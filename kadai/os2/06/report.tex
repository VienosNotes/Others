\documentclass[a4j,9pt]{jsarticle}
\usepackage{stdrep}
\usepackage{ascmac}	% required for `\screen' (yatex added)
\usepackage{fancybox}
\usepackage{geometry}
\usepackage{setspace}

%\setstretch{0.8}
\geometry{top=10mm,bottom=5mm}


\renewcommand{\thepage}{}

\title{情報科学類 オペレーティングシステムII 課題6}
\author{学籍番号 200911434 \\ 名前 青木大祐}

\lstset{
  language=C,
  basicstyle=\ttfamily\scriptsize,
  commentstyle=\textit,
  classoffset=1,
  keywordstyle=\bfseries,
  frame=trbl,
  framesep=5pt,
  showstringspaces=false,
  numbers=none,
  stepnumber=1,
  numberstyle=\tiny,
  tabsize=2,
  breaklines = true
}
\begin{document}
\maketitle
\setcounter{section}{5}
\section{デバイスドライバ、inb()とoutb()}
\setcounter{subsection}{600}

\subsection{CMOS RTC のデバイスドライバ}
\begin{screen}
{\ttfamily rtc-read-time.c}のプログラムでシステム・コールが実行されると、CMOS RTC
のデバイス・ドライバに含まれる関数が実行される。次のシステム・コールが
実行された時に、どんな関数が実行されるかを答えなさい。
{\ttfamily drivers/char/rtc.c}に含まれる関数の名前で答えなさい。
\begin{itemize}
 \item open() システム・コール(23行目):
 \item ioctl() システム・コール(28行目):
 \item close() システム・コール(36行目):
\end{itemize}
\end{screen}

\begin{description}
 \item[open()] rtc\_open
 \item[ioctl()] rtc\_ioctl
 \item[close()] rtc\_release
\end{description}

\subsection{copy\_to\_user()とmemcpy()}
\begin{screen}
関数{\ttfamily rtc\_do\_ioctl()}は、{\ttfamily copy\_to\_user()}を呼び出している。これは正しいプ
ログラムであるが、もしも{\ttfamily copy\_to\_user()}を{\ttfamily memcpy()}やポインタを操作
して直接ユーザ空間のメモリにアクセスすると問題がある。714行目から715行目
を、{\ttfamily memcpy()}を使って「間違ったプログラム」に書き換えなさい。
ただし、{\ttfamily copy\_to\_user()}は必ず成功するものとして、エラー処理を省略しなさい。

 以下の「/*空欄*/」を埋めなさい。

\begin{lstlisting}
memcpy( /*空欄(a)*/,/*空欄(b)*/,/*空欄(c)*/ );
return 0;
\end{lstlisting}
\end{screen}
\begin{lstlisting}
memcpy( arg, &wtime, sizeof wtime );
return 0;
\end{lstlisting}

\subsection{x86 CMOS RTCからの時間データの入力}
\begin{screen}
 次のプログラムはx86 CMOS RTCハードウェアから時間({\itshape hours})データ
 を読み出し変数{\ttfamily hh}に入れるプログラムの一部である。/*空欄(a)*/
 と/*空欄(b)*/を埋めて完成させなさい。
 \begin{lstlisting}
 unsigned char hh;
 outb( /*空欄(a)*/, 0x70 );
 hh = inb( /*空欄(b)*/ );
 \end{lstlisting}
\end{screen}

\begin{lstlisting}
unsigned char hh;
outb(4, 0x70 );
hh = inb( 0x71 );
\end{lstlisting}
\end{document}

