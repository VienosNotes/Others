\documentclass[a4j,9pt]{jsarticle}
\usepackage{stdrep}
\usepackage{ascmac}	% required for `\screen' (yatex added)
\usepackage{fancybox}
\usepackage{geometry}
\usepackage{setspace}

%\setstretch{0.8}
\geometry{top=10mm,bottom=5mm}


\renewcommand{\thepage}{}

\title{情報科学類 オペレーティングシステムII 課題5}
\author{学籍番号 200911434 \\ 名前 青木大祐}

\lstset{
  language=C,
  basicstyle=\ttfamily\scriptsize,
  commentstyle=\textit,
  classoffset=1,
  keywordstyle=\bfseries,
  frame=trbl,
  framesep=5pt,
  showstringspaces=false,
  numbers=none,
  stepnumber=1,
  numberstyle=\tiny,
  tabsize=2,
  breaklines = true
}
\begin{document}
\maketitle
\setcounter{section}{4}
\section{メモリ管理、アドレス空間、ページテーブル}
\setcounter{subsection}{500}

\subsection{/proc/PID/maps}
\begin{screen}
 /proc/PID/mapsの内容は、このページの中でどの構造体のリストを表示したも
 のと考えられるか。次の点を答えなさい。
\begin{itemize}
 \item リストの起点を保持している構造体の名前
 \item リストの起点を保持している構造体の中のフィールド名
 \item リストにつながれている構造体の名前
\end{itemize}
\end{screen}

\begin{description}
 \item[リストの起点を保持している構造体の名前] {\ttfamily mm\_struct}
 \item[リストの起点を保持している構造体の中のフィールド名] {\ttfamily mmap}
 \item[リストにつながれている構造体の名前] {\ttfamily vm\_area\_struct}
\end{description}

\subsection{1段のページテーブル}
\begin{screen}

 仮想アドレスのサイズが32ビット、1ページの大きさが4KBとする。 次の3ペー
 ジが割り当てられてしたとする。

\begin{itemize}
{\ttfamily 
 \item 0x00000000 から 0x00000fff まで
 \item 0x00001000 から 0x00001fff まで
 \item 0xfffff000 から 0xffffffff まで
}
\end{itemize}
1段のページテーブルを用いていた場合、ページテーブルに必要なメモリは何バイトになるか。ページテーブルの1エントリのバイトは、4バイトとする。なお、末端のページ・フレームに必要なメモリ(この場合は、3ページ、12KB)は、ページテーブルに必要なメモリではないので、計算に入れない。
\end{screen}
$4KB = 12bit$より$32 - 12 = 20bit$がページ内オフセット。1エントリ
$4byte$なので$4 * 2^{20} = 4MB$

\subsection{2段のページテーブル}
\begin{screen}
 問題(502)で、次のような2段のページテーブル(「x86のページ・テーブル」と同じ)を用いていたとする。
\begin{description}
 \item[1段目] 31..22ビット (上位10ビット)
 \item[2段目] 21..12ビット
 \item[オフセット] 下位12ビット (11..0ビット)
\end{description}

この時、ページテーブルに必要なメモリは何バイトになるか。ページテーブルの1エントリのバイトは、上位のページテーブルも下位のページテーブルも4バイトとする。
\end{screen}
それぞれの上位10bitに従って、1段目のテーブルは2つのエントリが、2段目にはページテー
ブルが2つ生成される。1つ4KBのテーブルが3つ生成されるので、合計12KB必要に
なる。
\end{document}

