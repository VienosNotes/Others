\documentclass[a4j,9pt]{jsarticle}
\usepackage{stdrep}
\usepackage{ascmac}	% required for `\screen' (yatex added)
\usepackage{fancybox}
\usepackage{geometry}
\usepackage{setspace}

%\setstretch{0.8}
\geometry{top=10mm,bottom=5mm}


\renewcommand{\thepage}{}

\title{情報科学類 オペレーティングシステムII 課題4}
\author{学籍番号 200911434 \\ 名前 青木大祐}

\lstset{
  language=C,
  basicstyle=\ttfamily\scriptsize,
  commentstyle=\textit,
  classoffset=1,
  keywordstyle=\bfseries,
  frame=trbl,
  framesep=5pt,
  showstringspaces=false,
  numbers=none,
  stepnumber=1,
  numberstyle=\tiny,
  tabsize=2,
  breaklines = true
}
\begin{document}
\maketitle
\setcounter{section}{3}
\section{メモリ管理、Buddyシステム、kmalloc、スラブアロケータ}
\setcounter{subsection}{400}
\subsection{struct page}
\begin{screen}
struct pageの大きさは、アーキテクチャやコンパイル時のオプションによって異なる。あるシステムで、struct pageの大きさが40バイトであったとする。そのシステムに、1GBのメモリが搭載されていた時、struct pageのために、何MBのメモリが使われるか。ページサイズは、4KBとする。
\end{screen}
$1024*1024*1024/(4*1024) * 40KB = 10MB$


\subsection{kmalloc()とkfree()}
\begin{screen}
以下は、ユーザ空間でメモリを割当て、利用し、開放するプログラムの一部である。
\begin{lstlisting}
   struct s1 *p;
   p = malloc( sizeof(struct s1) );
   use( p );
   free( p ); 
\end{lstlisting}
このプログラムを、カーネル内で動かすことを想定してkmalloc()とkfree()を使って書き換えなさい。ただし、gfp のフラグとしては、GFP\_KERNEL を使いなさい。
\begin{lstlisting}
利用
   struct s1 *p;
   /*回答*/
   use( p );
   /*回答*/ 
\end{lstlisting}
\end{screen}

\begin{lstlisting}
   struct s1 *p;
   p = kmalloc( sizeof(struct s1), GFP_KERNEL );
   use( p );
   kfree( p );
\end{lstlisting}

\subsection{スラブアロケータ}
\begin{screen}
問題(402) のプログラムを、スラブアロケータを使って書き換えなさい。すなわ
 ち、kmem\_cache\_create()、kmem\_cache\_alloc()、および、
 kmem\_cache\_free()を使って書き換えなさい。ただし、
 kmem\_cache\_create() の 第3引数のalignとしては、0を、第4引数のflagsと
 しては、SLAB\_PANIC、第5引数のコンストラクタとしては、NULLを指定しなさ
 い。

\begin{lstlisting}
初期化
   /*回答*/

利用
   struct s1 *p;
   /*回答*/
   use( p );
   /*回答*/
\end{lstlisting}
\end{screen}

\begin{lstlisting}
struct s1 *p;
kmem_cache_p = kmem_cache_create("hoge", sizeof(struct s1), 0, SLAB_PANIC, NULL);
p = kmem_cache_alloc(kmem_cache_p, GFP_KERNEL);
use( p );
kmem_cache_free(kmem_cache_p, p);
\end{lstlisting}
\end{document}

