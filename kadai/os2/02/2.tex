\documentclass[a4j,9pt]{jsarticle}
\usepackage{stdrep}
\usepackage{ascmac}	% required for `\screen' (yatex added)
\usepackage{fancybox}

\title{情報科学類 オペレーティングシステムII 課題2}
\author{学籍番号 200911434 \\ 名前 青木大祐}

\lstset{
  language=C,
  basicstyle=\ttfamily\scriptsize,
  commentstyle=\textit,
  classoffset=1,
  keywordstyle=\bfseries,
  frame=trbl,
  framesep=5pt,
  showstringspaces=false,
  numbers=none,
  stepnumber=1,
  numberstyle=\tiny,
  tabsize=2,
  breaklines = true
}

\begin{document}
\maketitle
\setcounter{section}{1}
\section{プロセスとtask\_struct構造体}
\setcounter{subsection}{200}
\subsection{Ready状態のプロセス}
\begin{screen}
オペレーティング・システムでは、「一般に」プロセスは、 実行待ち(Ready)、
 実行中(Running)、 待機中(Waiting、Blocked)という3つの 状態を持つ。
 Linux において、プロセスが Ready 状態であることを 示すために、task
 struct のフィールド state に、何という値を設定しているか。

\end{screen}

{\tt{TASK\_RUNNING}}

\subsection{pidとppid}
\begin{screen}
 次のプログラムをシェルから実行したとする。
\begin{lstlisting}
#include <stdio.h>
#include <unistd.h>

main() {
    pid_t pid, ppid;
    fork();
    pid = getpid();
    ppid = getppid();
    printf("hello: (pid=%d,ppid=%d)\n",pid, ppid);
}
\end{lstlisting}

以下の空欄(空欄A、空欄B、空欄C、空欄D)を埋めて、起こり得る結果を 1
 つ作りなさい。

\begin{lstlisting}
$ echo $$ 
1001
$ ./fork-printf  
hello: (pid=空欄A,ppid=空欄B)
hello: (pid=空欄C,ppid=空欄D)
$ 
\end{lstlisting}
ただし、PID としては、1001,1002,1003,1004 の中から選びなさい。 なお、答えは1通りではない。\\
上のプログラムをコンパイルして実行した結果を参考にして、回答してもよい。 ただし、PID としては、実行結果のものをそのまま使うのではなく、指定され たものを使いなさい。
\end{screen}
実際にviolaで動かした結果は以下のとおり。
\begin{lstlisting}
 % echo $$
4856
 % ./a.out   
hello: (pid=5415,ppid=5411)
hello: (pid=5411,ppid=4856)
\end{lstlisting}
これを踏まえて考えると、答えは次のようになると考えられる。

\begin{lstlisting}
hello: (pid=1003,ppid=1002)
hello: (pid=1002,ppid=1001)
\end{lstlisting}

\subsection{getuid()システムコール}

\begin{screen}
getuid() システム・コールを実装の概略を、今日の授業の範囲内で答えなさい。 利用する重要な変数、マクロ、構造体を列挙しなさい。そして、どのようにポ インタをたどっていくかを示しなさい。概略を記述するためには、簡易的なC 言語、日本語、または、英語を使いなさい。
なお、実際の getuid() システム・コールの実装は、名前空間の導入により複
 雑になっており、今日の授業の範囲を超えている。この課題では、実際のコー
 ドではなく、この授業の範囲内で答えなさい。(実際のコードをそのまま回答
 しても、得点を与えない。)
\end{screen}

\begin{itemize}
 \item  current\_uid()マクロを呼び出す
 \item その中で変数currentからフィールドcredを取り出し、その中のフィールドuidを返す
\end{itemize}
\end{document}

