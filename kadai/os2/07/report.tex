\documentclass[a4j,9pt]{jsarticle}
\usepackage{stdrep}
\usepackage{ascmac}	% required for `\screen' (yatex added)
\usepackage{fancybox}
\usepackage{geometry}
\usepackage{setspace}

%\setstretch{0.8}
\geometry{top=10mm,bottom=5mm}


\renewcommand{\thepage}{}

\title{情報科学類 オペレーティングシステムII 課題7}
\author{学籍番号 200911434 \\ 名前 青木大祐}

\lstset{
  language=C,
  basicstyle=\ttfamily\scriptsize,
  commentstyle=\textit,
  classoffset=1,
  keywordstyle=\bfseries,
  frame=trbl,
  framesep=5pt,
  showstringspaces=false,
  numbers=none,
  stepnumber=1,
  numberstyle=\tiny,
  tabsize=2,
  breaklines = true
}
\begin{document}
\maketitle
\setcounter{section}{6}
\section{割り込み処理}
\setcounter{subsection}{700}
\subsection{割り込みの利用}
\begin{screen}
キーボード、マウス、ネットワーク・カード等のデバイスからの入力では割り込みが多く使われている。割り込みを使う方法と割り込みを使わない方法を対比して、割り込みを使う方法の利点を説明しなさい。
\end{screen}
そのようなデバイスからはデータが入力されるタイミングがコンピュータ側から
制御できないため、割り込みを使わない場合はポーリングする必要がある。ポー
リングは間隔が長いと遅延が出るが、間隔が短いと負荷が高くなるため、このよ
うなデバイスには向かない。
一方割り込みを用いる場合は、デバイスがデータを準備してから任意のタイミン
グで入力を行えるため、ポーリングのような問題は発生せずオーバヘッドが小さ
くなる。

\subsection{x86 CMOS Real-Time Clock の割り込みハンドラ}
\begin{screen}
x86 CMOS Real-Time Clock の割り込みハンドラを関数名で答えなさい。その関
 数の引数と結果を、簡単に説明しなさい。
\end{screen}
\begin{description}
 \item[関数名] rtc\_interrupt
 \item[引数] \begin{description}
              \item[int irq] 割り込み番号
              \item[void *dev\_id] \mbox{} デバイス番号。ひとつの割り込みが複数デ
                         バイスで共有されるときに、デバイスを区別するた
                         めの値。
             \end{description}
\end{description}

\subsection{x86 CMOS Real-Time Clock の割り込みハンドラの呼び出し}
\begin{screen}
 今日の資料の中で、x86 CMOS Real-Time Clock の割り込みハンドラを呼び出し
 ていると思われる関数と行数を答えなさい。
\end{screen}

\begin{description}
 \item[mod\_timer()] 263
 \item[handle\_IRQ\_event()] 375
\end{description}

\end{document}

