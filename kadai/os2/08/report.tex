\documentclass[a4j,9pt]{jsarticle}
\usepackage{stdrep}
\usepackage{ascmac}	% required for `\screen' (yatex added)
\usepackage{fancybox}
\usepackage{geometry}
\usepackage{setspace}

%\setstretch{0.8}
\geometry{top=10mm,bottom=5mm}


\renewcommand{\thepage}{}

\title{情報科学類 オペレーティングシステムII 課題8}
\author{学籍番号 200911434 \\ 名前 青木大祐}

\lstset{
  language=C,
  basicstyle=\ttfamily\scriptsize,
  commentstyle=\textit,
  classoffset=1,
  keywordstyle=\bfseries,
  frame=trbl,
  framesep=5pt,
  showstringspaces=false,
  numbers=none,
  stepnumber=1,
  numberstyle=\tiny,
  tabsize=2,
  breaklines = true
}
\begin{document}
\maketitle
\setcounter{section}{7}
\section{割り込みの後半部、Softirq、Tasklet、Work Queue}
\setcounter{subsection}{800}
\subsection{割り込み後半の処理}
\begin{screen}
 割り込み処理を、前半(top half)と後半(bottom half)に分ける理由を簡単に説明しなさい。
\end{screen}
即座にデバイスに応答を返したいなど、割り込み処理中に他の割り込みを受
け入れたくない場合がある。しかし、その処理が全て終わるまで他のデバイ
スを待たせる訳にも行かないので、時間的制約のある処理だけを切り出して割り
込み禁止の前半部として処理を行い、そうでない部分は他の割り込みを受け入れ
ながら時間のあるときに後半部として処理する。

\subsection{Taskletの初期化}
\begin{screen}
{\itshape Tasklet}を使って次の関数{\ttfamily f()}を、割り込み処理の後半で呼び出したい。
\begin{lstlisting}
 void f(int arg1, int arg2) {
   省略;
 }
\end{lstlisting}
これを実現するために、どのような{\itshape Tasklet}のハンドラと初期化コードを書けばよいか。以下の空欄を埋めなさい。
\begin{lstlisting}
 void tasklet_handler(unsigned long data) { /* Tasklet ハンドラ */
    int arg1, arg2;
    arg1 = 省略;
    arg2 = 省略;
    /*空欄(a)*/
    その他の仕事;
 }

DECLARE\_TASKLET(/*空欄(b)*/, /*空欄(c)*/, 0); /* 構造体の初期化 */
\end{lstlisting}
注意: 構造体の名前は、次の問題の解答で利用する。それらしいものを付けなさい。
\end{screen}

\begin{lstlisting}
 void tasklet_handler(unsigned long data) { /* Tasklet ハンドラ */
    int arg1, arg2;
    arg1 = 省略;
    arg2 = 省略;
    h();)
    その他の仕事;
 }

DECLARE\_TASKLET(tasklet_h, tasklet_handler, 0); /* 構造体の初期化 */
\end{lstlisting}

\subsection{ハンドラの実行}
\begin{screen}
次のコードは、割り込みの前半部分(ハードウェアの割り込み)の一部である。割
 り込み処理の後半で、問題(802)で定義した{\itshape Tasklet}のハンドラを呼
 ぶように、空欄を埋めなさい。
\begin{lstlisting}
 
 irqreturn_t irq_handler(int irq, void *dev) {
    /*空欄(d)*/
    return IRQ_HANDLED;
 }
\end{lstlisting}
\end{screen}

\begin{lstlisting}
irqreturn_t irq_handler(int irq, void *dev) {
    tasklet_schedule( &tasklet_h );
    return IRQ_HANDLED;
}
\end{lstlisting}
\end{document}

