\documentclass[a4j,9pt]{jsarticle}
\usepackage{stdrep}
\usepackage{ascmac}	% required for `\screen' (yatex added)
\usepackage{fancybox}

\title{情報科学類 オペレーティングシステムII 課題1}
\author{学籍番号 200911434 \\ 名前 青木大祐}


\begin{document}
\maketitle
\section{システムコールとLinuxカーネルのソース}
\setcounter{subsection}{100}
\subsection{システム・コールとライブラリ関数の共通点}

\begin{screen}
C言語でプログラムを作成する場合、システム・コールとライブラリ関数の共通点を述べなさい。
\end{screen}

両方ともC言語からは関数呼び出しの形で利用できること。

\subsection{システム・コールとライブラリ関数の相違点}

\begin{screen}
システム・コールとライブラリ関数の相違点を述べなさい。
\end{screen}

\begin{itemize}
 \item システムコール
       \begin{itemize}
        \item カーネルのプログラムを直接的に利用する
        \item 最小を目指して設計する
        \item プロセスの外に働きかける
       \end{itemize}
 \item ライブラリ
       \begin{itemize}
        \item カーネルの機能を使わない
        \item 便利さを目指す
        \item システムコールの助けがなければメモリの内容を書き換えるだけ
       \end{itemize}
\end{itemize}

\subsection{chdir()システムコールの引数と結果}

\begin{screen}
このシステム・コールを処理する関数がカーネルの中でどのように定義されているか、その概略(引数と結果の宣言)を示しなさい。関数の内容は空でよい。マクロを利用しても利用しなくてもどちらでもよい。
\end{screen}

\subsubsection*{実装}
\begin{lstlisting}
asmlinkage long sys_chdir(const char __user * filename) {
 省略
}
\end{lstlisting}

\subsubsection*{引数と結果}
\begin{itemize}
 \item 引数 
       \begin{description}
        \item[const char \_\_user * filename] 移動するパス名
       \end{description}
 \item 返り値
       \begin{description}
        \item[long] 成功した場合は0, 失敗した場合はエラーコードを返す
       \end{description}
\end{itemize}
\end{document}

