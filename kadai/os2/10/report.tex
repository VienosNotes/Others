\documentclass[a4j,9pt]{jsarticle}
\usepackage{stdrep}
\usepackage{ascmac}	% required for `\screen' (yatex added)
\usepackage{fancybox}
\usepackage{geometry}
\usepackage{setspace}

%\setstretch{0.8}
\geometry{top=10mm,bottom=5mm}


\renewcommand{\thepage}{}

\title{情報科学類 オペレーティングシステムII 課題10}
\author{学籍番号 200911434 \\ 名前 青木大祐}

\lstset{
  language=C,
  basicstyle=\ttfamily\scriptsize,
  commentstyle=\textit,
  classoffset=1,
  keywordstyle=\bfseries,
  frame=trbl,
  framesep=5pt,
  showstringspaces=false,
  numbers=none,
  stepnumber=1,
  numberstyle=\tiny,
  tabsize=2,
  breaklines = true
}
\begin{document}
\maketitle
\setcounter{section}{9}
\section{ファイルシステム}
\setcounter{subsection}{1000}
\subsection{オブジェクト指向}
\begin{screen}
Linux におけるファイルシステムの実装では、C言語であるが、オブジェクト指
 向的な考え方が取り入れられている。どの部分で取り入れられているか、一例
 を示しなさい。次の項目を埋めなさい。
\begin{itemize}
 \item カーネル内の構造体の名前:
 \item 内部の変数(インスタンス変数)の例:
 \item その変数の役割:
 \item 内部の変数を操作するための公開された手続き(インスタンス・メソッド)の例:
 \item その手続きの役割:
\end{itemize}
\end{screen}
\begin{itemize}
 \item カーネル内の構造体の名前:
       \begin{lstlisting}
        struct inode
       \end{lstlisting}
 
 \item 内部の変数(インスタンス変数)の例:
       \begin{lstlisting}
        inode->i_count
       \end{lstlisting}
 \item その変数の役割:
       \begin{itemize}
        \item 参照カウンタ
       \end{itemize}
 \item 内部の変数を操作するための公開された手続き(インスタンス・メソッ
       ド)の例:
       \begin{lstlisting}
        inode->i_op->create
       \end{lstlisting}
 \item その手続きの役割:
       \begin{itemize}
        \item inodeと結び付けられたファイルを新しく作る
       \end{itemize}
\end{itemize}

\subsection{struct fileの役割}
\begin{screen}
Linux カーネルの中で、ファイルを表現するためのオブジェクトとしてstruct
 inodeとstruct fileがある。struct inodeだけでも、ファイルの操作(読
 み、書き、属性変更)では十分と思えるが、struct file も使われている。
 struct file の役割を1つ選んで簡単に説明しなさい。
\end{screen}
ディスクへの書き込みを効率化するために、ファイルの内容をメモリ上にマップ
する際に使われる。

\subsection{write()システムコール}
\begin{screen}
 
 次の関数は、write() システム・コールを実装している vfs\_write() の一部で ある。空欄を埋めなさい。
\begin{lstlisting}
  ssize_t vfs_write(struct file *file, const char __user *buf, size_t count, loff_t *pos)
 {
 ....
        ret = rw_verify_area(WRITE, file, pos, count);
                if (file->f_op->/*空欄(a)*/)
                        ret = file->f_op->/*空欄(b)*/(/*空欄(c)*/, buf, count, pos);
 ...
        }
        return ret;
 }
\end{lstlisting}
\end{screen}
\newpage
\begin{lstlisting}
  ssize_t vfs_write(struct file *file, const char __user *buf, size_t count, loff_t *pos)
 {
 ....
        ret = rw_verify_area(WRITE, file, pos, count);
                if (file->f_op->write)
                        ret = file->f_op->write(file, buf, count, pos);
 ...
        }
        return ret;
 }
\end{lstlisting}
\end{document}

