\documentclass[a4j,9pt]{jsarticle}
\usepackage{stdrep}
\usepackage{ascmac}	% required for `\screen' (yatex added)
\usepackage{fancybox}

\title{加速器と最先端科学 第3回レポート(第6回講義)}
\author{学籍番号 200911434 \\ 名前 青木大祐}

\begin{document}
\maketitle

\begin{screen}
講義を聞いて印象に残った検出器の名前をひとつ以上あげて、その放射線検出の
 原理と物質科学研究における使い道について、レポートにまとめなさい。
\end{screen}

\subsection*{シンチレーション検出器}
放射線が当たると励起し蛍光を発する物質(シンチレータ)を用い、その蛍光を光電子増倍管を用いて電
気エネルギーに変換し、これを測定する。シンチレータにはNaI(Tl)のような無
機結晶などが用いられ、他の方式に比べて安価に製造することができる。物質科
学研究における使い道としては分光器が挙げられ、ガンマ線のスペクトル解析な
どに用いられている。
\end{document}



