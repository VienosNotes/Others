\documentclass[a4j,9pt]{jsarticle}
\usepackage{stdrep}
\usepackage{ascmac}	% required for `\screen' (yatex added)
\usepackage{fancybox}

\title{情報科学概論I レポート}
\author{学籍番号 200911434 \\ 名前 青木大祐}


\begin{document}
\maketitle
\setcounter{section}{8}
\section{課題9}
\vspace{6cm}
\section{課題10}
取り外し可能なメディアは、本体の物理的な大きさに関わらず容量をスケール出来
る。ログの保存など、同時に複数部分の読み書きをしない運用においては、ほぼ
無限に容量を増やすことができるのが利点である。

\section{課題11}
\begin{description}
 \item[RAID 0] 複数のディスクにデータを分散して保存(ストライピング)し、読み書きの効率
            を高めたもの。耐障害性はない。
 \item[RAID 1] 複数のディスクに同じデータを保存(ミラーリング)し、耐障
            害性を高めたもの。RAIDドライバが一つしかない場合は、ドライバ
            が壊れた時にディスクにアクセスできなくなる。アクセス速度を向
            上させることもできる。
 \item[RAID 10] \textbf{RAID 1}でミラーリングされたドライブ群を
            \textbf{RAID 0}でストライピングした運用。ディスクが最低4ドラ
            イブ必要になるが、両方の恩恵を受けられる。
 \item[RAID 5] 
            データ本体と、パリティ(誤り訂正符号データ)を複数ディスクに
            分散して保存する。最低でも3台のドライブが必要で、2つ以上が同
            時に故障するとデータが読み出せない。書き込みが低速である。
 \item[RAID 6] 
            \textbf{RADI 5}のようなパリティを2種類作ることで、最大2台ま
            での障害に耐えることが出来る。\textbf{RAID 5}に比べて、同じ
            容量を保存するのに1台多くディスクが必要になる。更に書き込み
            が低速。

\end{description}

\section{課題12}
\begin{description}
 \item[MRAM] 磁気抵抗メモリ。DRAMのキャパシタ部分を磁気トンネル接合素子に置き換えた構造。
            DRAMと違って不揮発性という利点があるが、外部からの強磁界によっ
            て記憶が破壊される可能性がある。SRAMと同等の速度。
 \item[FeRAM] 強誘電体メモリ。仕組み自体はDRAMと同じであるが、キャパシタ
            が常誘電体ではなく強誘電体であるため不揮発性であり、またリフ
            レッシュ操作が不必要となるため消費電力が小さい。
 \item[PRAM] 相変化メモリ。FRAMのキャパシタ部分を相変化膜に置き換えた構
            造。不揮発性であり、生産には既存設備の流用が可能。DRAMより高
            速で寿命が長い。
 \item[ReRAM] 抵抗変化型メモリ。電圧印加による電気抵抗の大きな変化を利用
            して書き込みを行い、消費電力が小さい。単純な構造のため高密度
            化が可能であり、また電気抵抗の変化率が大きいため、多値化も可
            能である。速度はDRAMと同等。

\end{description}
\end{document}



