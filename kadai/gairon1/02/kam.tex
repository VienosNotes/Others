\documentclass[a4j,9pt]{jsarticle}


\usepackage{fancybox}
\usepackage{url}
\usepackage[dvipdfmx]{graphicx}
\usepackage{epsfig}
\usepackage{amsmath}
\usepackage{amssymb}
\usepackage{times}
\usepackage{ascmac}
\usepackage{here}
\usepackage{pxfonts}
\usepackage{listings, jlisting}

%\renewcommand{\lstlistingname}{
\lstset{
  basicstyle=\ttfamily\scriptsize,
  commentstyle=\textit,
  classoffset=1,
  keywordstyle=\bfseries,
  frame=trbl,
  framesep=5pt,
  showstringspaces=false,
%  numbers=left,
  stepnumber=1,
  numberstyle=\tiny,
  tabsize=2,
  breaklines = true,
  language=perl
}

\usepackage{geometry}
\geometry{left=25mm,right=25mm,top=0mm,bottom=25mm}

\setlength{\parskip}{0em}
\setlength{\topsep}{0em}
\setlength{\columnsep}{3zw}

\usepackage{setspace}


\title{情報科学概論I レポート}
\author{学籍番号 200911434 \\ 名前 青木大祐}


\begin{document}
\maketitle
\section*{マージソートの図示}
マージソートの手順は、以下のプログラムによって図示される。
\lstinputlisting{merge.pl}

\begin{lstlisting}
1,2,2,3,4,6,7,8,9,10

[10, 2, 8, 9, 2, 1, 4, 7, 6, 3]
[10, 2, 8, 9, 2][1, 4, 7, 6, 3]
[10, 2, 8][9, 2][1, 4, 7][6, 3]
[10, 2][8][9][2][1, 4][7][6][3]
{2, 10}{8}{9}{2}{1, 4}{7}{6}{3}
{2, 8, 10}{2, 9}{1, 4, 7}{3, 6}
{2, 2, 8, 9, 10}{1, 3, 4, 6, 7}
{1, 2, 2, 3, 4, 6, 7, 8, 9, 10}
\end{lstlisting}
\end{document}



