\documentclass[a4paper,9pt]{jarticle}
\usepackage{url}
\usepackage[dvipdfmx]{graphicx}
\usepackage{epsfig}
\usepackage{amsmath}
\usepackage{amssymb}
\usepackage{times}
\usepackage{ascmac}
\usepackage{here}
\usepackage{txfonts}
\usepackage{listings}

\renewcommand{\lstlistingname}{リスト}
\lstset{%language=Ocaml,
  basicstyle=\ttfamily\scriptsize,
  commentstyle=\textit,
  classoffset=1,
  keywordstyle=\bfseries,
  frame=tRBl,
  framesep=5pt,
  showstringspaces=false,
  numbers=left,
  stepnumber=1,
  numberstyle=\tiny,
  tabsize=2
}

 \lstdefinelanguage{CASL2}{
   morekeywords={C, C++, Ruby},
   morecomment=[l]{;},
   morestring=[b]",%"
 }

\setlength{\oddsidemargin}{-.2in}
\setlength{\evensidemargin}{-.2in}
\setlength{\topmargin}{-4em}
\setlength{\textwidth}{6.5in}
\setlength{\textheight}{10in}
\setlength{\parskip}{0em}
\setlength{\topsep}{0em}
\setlength{\columnsep}{3zw}

\title{ソフトウェアサイエンス実験 S8 課題1-3}
\author{200911434 青木大祐}

\begin{document}
\maketitle
\newpage
\setcounter{section}{1}
\setcounter{subsection}{3}
\subsubsection{最大公約数}
ユークリッドの互除法を用いて最大公約数を計算する。引数のどちらかが負数だっ
た場合は絶対値を引数にして計算を行う。

\lstinputlisting{1_1.ml}

\begin{lstlisting}[caption=実行結果]
 #   - : int = 5
 # - : int = 22
\end{lstlisting}
正しく計算できていることが分かる。

\subsubsection{フィボナッチ数}
以下の関数はフィボナッチ数列の、引数に指定したn番目の数字を出力する。

\lstinputlisting{1_2.ml}

\begin{lstlisting}[caption=実行結果]
   #   - : int = 55
   # - : int = 6765
\end{lstlisting}
正しく計算できていることが分かる。
 
\subsubsection{クイックソート}
再帰を用いてクイックソートを実装した。リストの先頭をpivotに指定したList.partitionで左右に分割し、それぞれに対して再帰的にクイックソートを適用していく。

\lstinputlisting{1_5.ml}
\begin{lstlisting}[caption=実行結果]
   #   - : int list = [1; 3; 5; 5; 7; 7; 13; 14; 32; 52]
\end{lstlisting}
これより正しくソートされていることがわかる。
\end{document}