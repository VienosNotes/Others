\documentclass[a4paper,9pt]{jarticle}
\usepackage{stdrep}

\title{ソフトウェアサイエンス実験 S8 課題3-1}
\author{200911434 青木大祐}

\begin{document}
\maketitle
\setcounter{section}{3}
\setcounter{subsection}{1}

\newpage
\subsubsection{環境を作る}
以下のソースコードを用いて環境を作った。
\begin{lstlisting}
let env = emptyenv();;
let env = ext env "x" (IntVal 1);;
let env = ext env "y" (BoolVal true);;
let env = ext env "z" (IntVal 5);;
\end{lstlisting}

出力は以下のとおり。正しく環境が作られている。
\begin{lstlisting}
#     val env : 'a list = []
# val env : (string * value) list = [("x", IntVal 1)]
# val env : (string * value) list = [("y", BoolVal true); ("x", IntVal 1)]
# val env : (string * value) list =
  [("z", IntVal 5); ("y", BoolVal true); ("x", IntVal 1)]
\end{lstlisting}


\subsubsection{変数を調べる}
以下のソースコードのとおり、$env$を検索した。
\begin{lstlisting}
let y = lookup "y" env;;
let w = lookup "w" env;;
\end{lstlisting}

出力は以下のとおり。$y$は登録されており$BoolVal\:true$が返る。また、$w$は
登録されていないため、例外が発生しているのが分かる。
\begin{lstlisting}
#   val y : value = BoolVal true
# Exception: Failure "unbound variable: w".
\end{lstlisting}

\subsubsection{変数の追加順}
$ext$関数では$(x,v) :: env$のようにリストの先頭に要素を追加している。ま
た、$lookup$関数は先頭から探索を行うため、後から追加した同名の変数が優先
的にヒットする。\\
よって、新しい値が生き残るといえる。
\end{document}

