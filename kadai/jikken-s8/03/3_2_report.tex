\documentclass[a4paper,9pt]{jarticle}
\usepackage{stdrep}

\title{ソフトウェアサイエンス実験 S8 課題3-2}
\author{200911434 青木大祐}

\begin{document}
\maketitle
\setcounter{section}{3}
\setcounter{subsection}{1}

\newpage
作成したeval3のソースコードを以下に示す。
\begin{lstlisting}
\begin{lstlisting}
(* eval2b : exp -> value *)

type exp =
  | IntLit of int
  | Plus of exp * exp 
  | Times of exp * exp
  | BoolLit of bool        (* 追加分; 真理値リテラル, つまり trueや false  *)
  | If of exp * exp * exp  (* 追加分; if-then-else式 *)
  | Eq of exp * exp        (* 追加分; e1 = e2 *)
  | Greater of exp * exp 
  | Var of string        
  | Let of string * exp * exp ;;

(* 値の型 *)
type value =
  | IntVal  of int          (* 整数の値 *)
  | BoolVal of bool         (* 真理値の値 *);;
 
let emptyenv () = [];;

let ext env x v = (x,v) :: env;;

let rec lookup x env =
   match env with
   | [] -> failwith ("unbound variable: " ^ x)
   | (y,v)::tl -> if x=y then v 
                  else lookup x tl;;

(* eval3 : exp -> (string * value) list -> value *)
(* let と変数、環境の導入 *)
let rec eval3 e env =           (* env を引数に追加 *)
  let binop f e1 e2 env =       (* binop の中でも eval3 を呼ぶので env を追加 *)
    match (eval3 e1 env, eval3 e2 env) with
    | (IntVal(n1),IntVal(n2)) -> IntVal(f n1 n2)
    | _ -> failwith "integer value expected"
  in 
  match e with
  | Var(x)       -> lookup x env
  | IntLit(n)    -> IntVal(n)
  | BoolLit(b) -> BoolVal(b)
  | Plus(e1,e2)  -> binop (+) e1 e2 env     (* env を追加 *)
  | Times(e1,e2) -> binop ( * ) e1 e2 env   (* env を追加 *)
  | Eq(e1,e2) ->
      begin
	match (eval3 e1 env, eval3 e2 env) with
	  | (IntVal(n1),IntVal(n2)) -> BoolVal(n1=n2)
	  | (BoolVal(b1),BoolVal(b2)) -> BoolVal(b1=b2)
	  | _ -> failwith "wrong value"
      end
  | If(e1,e2,e3) ->
      begin
        match (eval3 e1 env) with          (* env を追加 *)
          | BoolVal(true)  -> eval3 e2 env   (* env を追加 *)
          | BoolVal(false) -> eval3 e3 env   (* env を追加 *)
          | _ -> failwith "wrong value"
      end
  | Let(x,e1,e2) -> 
      let env1 = ext env x (eval3 e1 env)
      in eval3 e2 env1
  | _ -> failwith "unknown expression";;
\end{lstlisting}

\subsubsection{動作の確認}
次のような例を与えて動作を確認した。

\begin{lstlisting}
eval3 (Let ("x", IntLit 1, (Plus (IntLit 2, Var "x")))) [];;
eval3 (Let ("x", IntLit 1, Let("y", IntLit 3, (Plus (Var "y", Var "x"))))) [];;
eval3 (Let ("x", BoolLit true, If(Eq(Var "x", BoolLit true), IntLit 1, IntLit 2))) [];;
\end{lstlisting}

実行結果は以下の通り。
\begin{lstlisting}
# - : value = IntVal 3
# - : value = IntVal 4
# - : value = IntVal 1
\end{lstlisting}
正しく実行できていることが分かる。

\subsubsection{letのネスト}
次のような敷で、let文をネストした際の動作を確認した。
\begin{lstlisting}
eval3 (Let ("x", IntLit 1, (Let ("x", IntLit 2, Var "x")))) [];;
\end{lstlisting}
実行結果は以下のとおり。
\begin{lstlisting}
#   - : value = IntVal 2
\end{lstlisting}

また、Ocamlで実行した結果は以下のとおり。同じように計算できていることが
分かる。
\begin{lstlisting}
# let x = 1 in let x = 2 in x;;
Warning 26: unused variable x.
- : int = 2
\end{lstlisting}


\end{document}

