\documentclass[a4paper,9pt]{jarticle}
\usepackage{stdrep}

\title{ソフトウェアサイエンス実験 S8 課題5-2}
\author{200911434 青木大祐}

\begin{document}
\maketitle
\setcounter{section}{5}
\setcounter{subsection}{2}

\newpage
\subsubsection{eval4の実装}
\begin{lstlisting}
let x = 1 in
  let f = fun y -> x + y in
    let x = 2 in
      f (x + 3)
\end{lstlisting}
上記のプログラムを、ミニOcamlのプログラムとして実行した。結果を以下に示
す。

\begin{lstlisting}
- : Syntax.value = Syntax.IntVal 6
\end{lstlisting}

\subsubsection{APPの変更}
上記のプログラムについて、$eval4$内の$env1$を$env$に変更した際の実行結果
を示す。

\begin{lstlisting}
- : Syntax.value = Syntax.IntVal 7
\end{lstlisting}

$env$はクロージャに保持されている環境ではなく、関数適用が行われた時点での
環境なので、これは動的束縛だといえる。

\subsubsection{factorialの定義}
ある数の階乗を求めるには、ループもしくは再帰の構造が必要になる。ミニ
OCamlではループ構文がまだ用意されていないので、再帰を使うことを考える。
\\
しかしミニOCamlでは、関数抽象を行う時点での環境にその関数自体は含まれ
ていないため、$"unbound variable: fact"$のように未定義の変数であるという
ふうにエラーが出てしまう。\\
よって、ここまでの段階ではfactorialを定義することは不可能だと考えられる。

\subsubsection{評価順序}
$(e1+e2)$について、実際に評価される部分は以下の二行目である。

\begin{lstlisting}
  let binop f e1 e2 env =       (* binop の中でも eval4 を呼ぶので env を追加 *)
    match (eval4 e1 env ~mode, eval4 e2 env ~mode) with
    | (IntVal(n1),IntVal(n2)) -> IntVal(f n1 n2)
    | _ -> failwith "integer value expected"
  in 
\end{lstlisting}
また、Ocamlで以下のプログラムを実行すると次のような結果が得られる。
\begin{lstlisting}
#   (print_string "1"; 10) + (print_string "2"; 20);;
21- : int = 30

# (print_int 1, print_int 2);;
21- : unit * unit = ((), ())
\end{lstlisting}
この結果から、カンマ区切りの式は後者が先に評価されるので、
$(e1+e2)$についてはOCamlとミニOCamlで違いは無いと考えられる。\\

$(e1 e2)$という形の関数適用について、例として提示されたプログラムを実行
すると、以下の様な結果が得られた。

\begin{lstlisting}
#   (print_string "1"; (fun x -> x)) (print_string "2"; 20) ;;
21- : int = 20
\end{lstlisting}
これは、適用する関数本体より先に引数を評価していることを表している。しか
し、現時点でのミニOCamlの実装は以下のとおりである。

\begin{lstlisting}
  | App(e1,e2) ->
     begin
      match (eval4 e1 env) with
        | FunVal(x,body,env1) ->
            let arg = (eval4 e2 env) 
	    in eval4 body (ext env1 x arg)
        | _ -> failwith "function value expected"
     end
\end{lstlisting}
これを次のように変更することで、OCamlと同じ評価順序になるeval4関数が得ら
れる。

\begin{lstlisting}
  | App(e1,e2) ->
     begin
      let arg = (eval4 e2 env) in
      match (eval4 e1 env) with
        | FunVal(x,body,env1) ->
	    eval4 body (ext env1 x arg)
        | _ -> failwith "function value expected"
     end
\end{lstlisting}
\end{document}

