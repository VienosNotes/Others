\documentclass[a4paper,9pt]{jarticle}
\usepackage{stdrep}

\title{ソフトウェアサイエンス実験 S8 課題5-1}
\author{200911434 青木大祐}

\begin{document}
\maketitle
\setcounter{section}{5}
\setcounter{subsection}{1}

\newpage

\subsubsection{静的束縛と動的束縛}
以下のプログラムについて、静的束縛と動的束縛の場合についての計算過程を示
す。

\begin{lstlisting}
 let x = 10 in
  let f = fun y -> x + y in
    let y = f x in 
     let x = 20 in 
       f (x * y)
\end{lstlisting}

\paragraph{静的束縛の場合}

\begin{enumerate}
 \item 一番外側のスコープにおいて、x=10という束縛を有効にする
 \item 関数fの定義について、xは一番外側の束縛が有効になるので、f=10+yという束縛が有効になる
 \item 変数yについて、関数fの引数xに一番外側のxの束縛が適用されるので、y=20という束縛が有効になる
 \item 変数xについて、一番外側のxの束縛を上書きしてx=20という束縛が有効になる
 \item f=(x * y)について、内側のxの束縛とyの束縛が適用されるので、f(20
       * 20) = f(400)という適用になる
       \begin{enumerate}
        \item f=x+yについて、xは関数が定義された時点の10が有効
        \item yについては引数に渡されたy=400の束縛が有効になる
       \end{enumerate}
 \item fを適用した結果として410が返る
\end{enumerate}

\paragraph{動的束縛の場合}
\begin{enumerate}
 \item x=10という束縛を有効にする
 \item 関数fの定義について、f=x+yという束縛を有効にする
 \item 変数yについて、関数fにxを適用する、つまりx=10より10+10なのでy=20という束縛が有効になる
 \item 変数xについて、一番外側のxの束縛を上書きしてx=20という束縛が有効になる
 \item f(x * y)について、f(20 * 20) = f(400)という適用になる
       \begin{enumerate}
        \item f=x+yについて、xは最新の束縛である20が有効になる
        \item yについて、引数に渡されたy=400という束縛が有効になる
       \end{enumerate}

 \item fを適用した結果として420が返る
\end{enumerate}
\end{document}

