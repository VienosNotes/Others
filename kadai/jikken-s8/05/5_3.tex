\documentclass[a4paper,9pt]{jarticle}
\usepackage{stdrep}

\title{ソフトウェアサイエンス実験 S8 課題5-3}
\author{200911434 青木大祐}

\begin{document}
\maketitle
\setcounter{section}{5}
\setcounter{subsection}{3}

\newpage
\subsubsection{テストプログラムの実行}
作成した$eval6$関数の動作を確認するため、次のような入力を与えた。また、
このプログラムを動作させるために、$Minus$に関するパターンを$binop$を用い
て実装した。これは$Plus$と変わらないため省略する。

\begin{lstlisting}
run "let rec f x = x in 0";;
run "let rec f x = x in f 0";;
run "let rec f x = if x = 0 then 1 else 2 + f (x + (-1)) in f 1";;
run "let rec f x = if x = 0 then 1 else x * f (x + (-1)) in f 3";;
run "let rec f x = if x = 0 then 1 else x * f (x + (-1)) in f 5";;
\end{lstlisting}

実行結果を以下に示す。

\begin{lstlisting}
- : Syntax.value = Syntax.IntVal 0
- : Syntax.value = Syntax.IntVal 0
- : Syntax.value = Syntax.IntVal 3
- : Syntax.value = Syntax.IntVal 6
- : Syntax.value = Syntax.IntVal 120
\end{lstlisting}
正しく計算できていることが分かる。

\subsubsection{再帰関数の定義}
$n$番目のフィボナッチ数を求める$fib$関数をミニOCamlで実装した。

\begin{lstlisting}
run "let rec fib x = if x = 0 then 0 else x + fib(x - 1) in fib 10";;
# - : Syntax.value = Syntax.IntVal 55
\end{lstlisting}
\end{document}

