\documentclass[a4paper,9pt]{jarticle}
\usepackage{url}
\usepackage[dvipdfmx]{graphicx}
\usepackage{epsfig}
\usepackage{amsmath}
\usepackage{amssymb}
\usepackage{times}
\usepackage{ascmac}
\usepackage{here}
\usepackage{txfonts}
\usepackage{listings, jlisting}

\renewcommand{\lstlistingname}{リスト}
\lstset{%language=Ocaml,
  basicstyle=\ttfamily\scriptsize,
  commentstyle=\textit,
  classoffset=1,
  keywordstyle=\bfseries,
  frame=tRBl,
  framesep=5pt,
  showstringspaces=false,
  numbers=left,
  stepnumber=1,
  numberstyle=\tiny,
  tabsize=2
}

 \lstdefinelanguage{CASL2}{
   morekeywords={C, C++, Ruby},
   morecomment=[l]{;},
   morestring=[b]",%"
 }

\setlength{\oddsidemargin}{-.2in}
\setlength{\evensidemargin}{-.2in}
\setlength{\topmargin}{-4em}
\setlength{\textwidth}{6.5in}
\setlength{\textheight}{10in}
\setlength{\parskip}{0em}
\setlength{\topsep}{0em}
\setlength{\columnsep}{3zw}

\title{ソフトウェアサイエンス実験 S8 課題2-4}
\author{200911434 青木大祐}

\begin{document}
\maketitle
\setcounter{section}{2}
\setcounter{subsection}{4}

\newpage
以下に課題2-4のソースコードを示す。
\lstinputlisting{./2_4_1.ml}
\newpage

\subsubsection{テスト}

以下にテストコードと、その実行結果を示す。\\
\begin{lstlisting}
print_string "2.4.1 テスト";;
let _ = eval2 (IntLit 1);;
let _ = eval2 (IntLit 11);;
let _ = eval2 (Plus (IntLit 1, Plus (IntLit 2, IntLit 11)));;
let _ = eval2 (Times (IntLit 1, Plus (IntLit 2, IntLit 11)));;
let _ = eval2 (If (Eq(IntLit 2, IntLit 11),
                   Times(IntLit 1, IntLit 2),
                   Times(IntLit 1, Plus(IntLit 2,IntLit 3))));;
let _ = eval2 (Eq (IntLit 1, IntLit 1));;
let _ = eval2 (Eq (IntLit 1, IntLit 2));;
let _ = eval2 (Eq (BoolLit true, BoolLit true));;
let _ = eval2 (Eq (BoolLit true, BoolLit false));;
\end{lstlisting}

\begin{lstlisting}
#     2.4.1 テスト- : unit = ()
# - : value = IntVal 1
# - : value = IntVal 11
# - : value = IntVal 14
# - : value = IntVal 13
# - : value = IntVal 5
# - : value = BoolVal true
# - : value = BoolVal false
# - : value = BoolVal true
# - : value = BoolVal false
\end{lstlisting}
正しく計算できていることがわかる。

\subsubsection{エラーを起こす適当な例}

\begin{lstlisting}
print_string "2.4.2 適当な例でエラーを起こす";;
eval2 (Plus (IntLit 10, BoolLit(true)));;
eval2 (If ((IntLit 1), (IntLit 2), (IntLit 3)));;
\end{lstlisting}
このコードを実行すると、以下のように例外が発生する。

\begin{lstlisting}
#   2.4.2 適当な例でエラーを起こす- : unit = ()
# Exception: Failure "integer values expected".
# Exception: Failure "wrong value".
\end{lstlisting}
どちらも引数として適合しない型の値を渡しているため、パターンマッチに用意
されている例外が発生する。

\subsubsection{整数と真理値の比較}
\begin{lstlisting}
print_string "2.4.3 整数と真理値をeqで比較する";;
eval2( Eq(IntLit 1, BoolLit true));;
\end{lstlisting}
実行結果は以下のとおり。
\begin{lstlisting}
#   2.4.3 整数と真理値をeqで比較する- : unit = ()
# Exception: Failure "wrong value".
\end{lstlisting}
35行目、36行目のどちらにもマッチしないので、45行目の例外が発生する。

\newpage

\subsubsection{Greaterの実装}

\begin{lstlisting}
print_string "2.4.4 Greaterを実装する";;
eval2 (Greater (IntLit 10, IntLit 50));;
eval2 (Greater (IntLit 10, IntLit 5));;
\end{lstlisting}
実行結果は以下のとおり。
\begin{lstlisting}
#   2.4.4 Greaterを実装する- : unit = ()
# - : value = BoolVal false
# - : value = BoolVal true
\end{lstlisting}
正しく計算できている。

\end{document}
