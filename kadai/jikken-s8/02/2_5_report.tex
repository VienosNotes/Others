\documentclass[a4paper,9pt]{jarticle}
\usepackage{url}
\usepackage[dvipdfmx]{graphicx}
\usepackage{epsfig}
\usepackage{amsmath}
\usepackage{amssymb}
\usepackage{times}
\usepackage{ascmac}
\usepackage{here}
\usepackage{txfonts}
\usepackage{listings, jlisting}

\renewcommand{\lstlistingname}{リスト}
\lstset{%language=Ocaml,
  basicstyle=\ttfamily\scriptsize,
  commentstyle=\textit,
  classoffset=1,
  keywordstyle=\bfseries,
  frame=tRBl,
  framesep=5pt,
  showstringspaces=false,
  numbers=left,
  stepnumber=1,
  numberstyle=\tiny,
  tabsize=2
}

 \lstdefinelanguage{CASL2}{
   morekeywords={C, C++, Ruby},
   morecomment=[l]{;},
   morestring=[b]",%"
 }

\setlength{\oddsidemargin}{-.2in}
\setlength{\evensidemargin}{-.2in}
\setlength{\topmargin}{-4em}
\setlength{\textwidth}{6.5in}
\setlength{\textheight}{10in}
\setlength{\parskip}{0em}
\setlength{\topsep}{0em}
\setlength{\columnsep}{3zw}

\title{ソフトウェアサイエンス実験 S8 課題2-5}
\author{200911434 青木大祐}

\begin{document}
\maketitle
\setcounter{section}{2}
\setcounter{subsection}{5}

\newpage

\subsubsection{内部の関数が呼べないことを確かめる}
以下にソースコードを示す。
\begin{lstlisting}
type exp =
  | IntLit of int
  | Plus of exp * exp 
  | Times of exp * exp
  | BoolLit of bool        (* 追加分; 真理値リテラル, つまり trueや false  *)
  | If of exp * exp * exp  (* 追加分; if-then-else式 *)
  | Eq of exp * exp        (* 追加分; e1 = e2 *)
  | Greater of exp * exp ;;

(* 値の型 *)
type value =
  | IntVal  of int          (* 整数の値 *)
  | BoolVal of bool         (* 真理値の値 *);;

let rec eval2b e =
  let binop f e1 e2 =
    match (eval2b e1, eval2b e2) with
      | (IntVal(n1),IntVal(n2)) -> IntVal(f n1 n2)
      | _ -> failwith "integer values expected"
  in 
  match e with
    | IntLit(n)    -> IntVal(n)
    | Plus(e1,e2)  -> binoop (+) e1 e2
    | Times(e1,e2) -> binop ( * ) e1 e2
    | _ -> failwith "unknown expression";;

binop (+) (IntLit 10) (IntLit 20);;
\end{lstlisting}

これを実行すると、以下のようなエラーが出力され、binop関数が呼び出せないことが分かる。
\begin{lstlisting}
#         type value = IntVal of int | BoolVal of bool
#                       val eval2b : exp -> value = <fun>
#   Characters 1-6:
  binop (+) (IntLit 10) (IntLit 20);;
  ^^^^^
Error: Unbound value binop
# 
\end{lstlisting}

\subsubsection{外から使えない理由}
内部で定義した関数が外から呼び出せない利点として、名前を汚染しないことが
考えられる。例えばこのプログラムを外部から参照して利用するとして、偶然
binopという関数を定義してしまうと、名前が衝突してしまいコンパイルできず、
これを解消するために別の(最適でない)名前を考える必要が生じる。しかし、内
部で定義した関数の名前が外にもれないことで、この問題は解決することが出来
る。


\subsubsection{binopの型}
binopを外に出した形のソースコードは以下のとおり。
\begin{lstlisting}
(* eval2b : exp -> value *)

type exp =
  | IntLit of int
  | Plus of exp * exp 
  | Times of exp * exp
  | BoolLit of bool        (* 追加分; 真理値リテラル, つまり trueや false  *)
  | If of exp * exp * exp  (* 追加分; if-then-else式 *)
  | Eq of exp * exp        (* 追加分; e1 = e2 *)
  | Greater of exp * exp ;;

(* 値の型 *)
type value =
  | IntVal  of int          (* 整数の値 *)
  | BoolVal of bool         (* 真理値の値 *);;

let rec eval2b e =
  let binop f e1 e2 =
    match (eval2b e1, eval2b e2) with
      | (IntVal(n1),IntVal(n2)) -> IntVal(f n1 n2)
      | _ -> failwith "integer values expected"
  in 
  match e with
    | IntLit(n)    -> IntVal(n)
    | Plus(e1,e2)  -> binop (+) e1 e2
    | Times(e1,e2) -> binop ( * ) e1 e2
    | _ -> failwith "unknown expression";;

eval2b (Plus ((IntLit 10), (IntLit 20)));;

let binop f e1 e2 =
  match (eval2b e1, eval2b e2) with
    | (IntVal(n1),IntVal(n2)) -> IntVal(f n1 n2)
    | _ -> failwith "integer values expected";;
        
binop (+) (IntLit 10) (IntLit 20);;
\end{lstlisting}
これを実行すると、以下の様な結果が得られる。

\begin{lstlisting}
#         type value = IntVal of int | BoolVal of bool
#                       val eval2b : exp -> value = <fun>
#   - : value = IntVal 30
#         val binop : (int -> int -> int) -> exp -> exp -> value = <fun>
#   - : value = IntVal 30
# 
\end{lstlisting}

これより、binopの型は$(int -> int -> int) -> exp -> exp -> value =
<fun>$であることが分かる。
\end{document}

