\documentclass[a4paper,9pt]{jarticle}
\usepackage{url}
\usepackage[dvipdfmx]{graphicx}
\usepackage{epsfig}
\usepackage{amsmath}
\usepackage{amssymb}
\usepackage{times}
\usepackage{ascmac}
\usepackage{here}
\usepackage{txfonts}
\usepackage{listings}

\renewcommand{\lstlistingname}{リスト}
\lstset{%language=Ocaml,
  basicstyle=\ttfamily\scriptsize,
  commentstyle=\textit,
  classoffset=1,
  keywordstyle=\bfseries,
  frame=tRBl,
  framesep=5pt,
  showstringspaces=false,
  numbers=left,
  stepnumber=1,
  numberstyle=\tiny,
  tabsize=2
}

 \lstdefinelanguage{CASL2}{
   morekeywords={C, C++, Ruby},
   morecomment=[l]{;},
   morestring=[b]",%"
 }

\setlength{\oddsidemargin}{-.2in}
\setlength{\evensidemargin}{-.2in}
\setlength{\topmargin}{-4em}
\setlength{\textwidth}{6.5in}
\setlength{\textheight}{10in}
\setlength{\parskip}{0em}
\setlength{\topsep}{0em}
\setlength{\columnsep}{3zw}

\title{ソフトウェアサイエンス実験 S8 課題2-2}
\author{200911434 青木大祐}

\begin{document}
\maketitle
\setcounter{section}{2}
\setcounter{subsection}{2}

\newpage

\subsubsection{型を見る}
以下の式について、REPLで実行して型を確かめた。
\begin{lstlisting}
IntLit 1;;
Plus(IntLit 1, IntLit 3);;
\end{lstlisting}

\begin{lstlisting}[caption=実行結果]
 #     - : exp = IntLit 1
 # - : exp = Plus (IntLit 1, IntLit 3)
\end{lstlisting}
exp型の式になっていることが分かる。

\subsubsection{引き算と割り算の追加}
以下のように、exp型に引き算と割り算を追加した。
\begin{lstlisting}
 type exp =
  | IntLit of int         (* リテラル *)
  | Plus of exp * exp     (* 足し算 *)
  | Times of exp * exp    (* かけ算 *)
  | Sub of exp * exp      (* 引き算 *)
  | Div of exp * exp      (* 割り算 *);;
\end{lstlisting}

\subsubsection{式+2}
式Eを受け取り、Plus(E + IntLit2)を返す関数を以下のように実装し、式sample
に対して適用した。
\begin{lstlisting}
let func (e:exp) =
  Plus(e, IntLit 2);;

let sample:exp = Div(Plus(IntLit 1, Times(IntLit (-2), IntLit 5)), Sub(IntLit 4, IntLit (-3)));;

func sample;;
\end{lstlisting}

\begin{lstlisting}[caption=実行結果]
 Plus
 (Div (Plus (IntLit 1, Times (IntLit (-2), IntLit 5)),
   Sub (IntLit 4, IntLit (-3))),
 IntLit 2)
\end{lstlisting}

\subsubsection{絶対値}
式の中に出現する整数リテラルを絶対値にする関数を作成した。 再帰的に式を辿って行き、整数リテラルをabs関数で絶対値にする。
\begin{lstlisting}
let rec abs_exp (e:exp) :exp = 
  match e with
    | IntLit i -> IntLit (abs i)
    | Plus(a, b) -> Plus((abs_exp a), (abs_exp b))
    | Times(a, b) -> Times((abs_exp a), (abs_exp b))
    | Sub(a, b) -> Sub((abs_exp a), (abs_exp b))
    | Div(a, b) -> Div((abs_exp a), (abs_exp b));;

abs_exp sample;;
\end{lstlisting}

\begin{lstlisting}[caption=実行結果]
   Div (Plus (IntLit 1, Times (IntLit 2, IntLit 5)), Sub (IntLit 4, IntLit 3)) 
\end{lstlisting}
\end{document}