\documentclass[a4paper,9pt]{jsarticle}
\usepackage{stdrep}

\title{ソフトウェアサイエンス実験 S8 最終レポート}
\author{200911434 青木大祐}

\begin{document}
\maketitle
\newpage

\section{概要}
本実験では、関数プログラミング言語Ocamlを用いて、Ocaml自身のサブセットで
ある「ミニOcaml言語(以下「ミニOcaml」)」のインタプリタおよびコンパイラを作成した。

\section{Ocamlについて}
まず最初に、実験に用いるOcamlについて理解を深めるため、関数プログラミン
グの技法を学んだ。具体的には、再帰やパターンマッチを用いてEuclidの互除法から最大公約数を
求める関数や、Fibonacci数を求める関数を記述した。例を以下に示す。


\begin{lstlisting}
let rec gcd (x, y) =   
  if x <= 0 || y <= 0 then gcd (abs x, abs y)
  else if x = y then x
  else if x > y then gcd(x-y, y)
  else gcd(x, y-x);;

let rec fib n = 
  match n with
      1 -> 1
    | 2 -> 1
    | n -> fib(n - 2) + fib(n - 1);;
\end{lstlisting}

\section{ミニOcaml言語}
前章の内容を終えた後、本題である「ミニOcamlインタプリタの実装」に着
手した。本実験で扱うミニOcamlの構文は以下のとおり\footnote{実験テキスト
(http://logic.cs.tsukuba.ac.jp/~kam/jikken/mini.html)より引用}である。

\begin{lstlisting}
e ::= x                   変数
  | let x = e in e        let式
  | let rec f x = e in e  let-rec式
  | fun x -> e            関数
  | e e	                  関数適用
  | true                  真理値リテラル(定数)
  | false                 真理値リテラル(定数)
  | if e then e else e    if式
  | n                     自然数リテラル (n は自然数 0, 1, 2, ...)
  | - e                   整数演算(符号の反転)
  | e + e                 整数演算(足し算)
  | e * e                 整数演算(かけ算)
  | e / e                 整数演算(割り算)
  | e = e                 等しさ(整数と真理値)
  | e > e                 大小比較(整数)
  | e < e                 大小比較(整数)
\end{lstlisting}
このミニOcamlを実装するにあたって、入力されるプログラムを扱う内部的な表
現として{\bf{式}}($exp$)という構造を考える。プログラムは式からなり、また式は一つ
ないしは複数の部分的な式から成り立っている。この再帰的な構造を表現するた
め、Ocamlのヴァリアント型を用いて代数的データ構造を実現することにする。\\

上記のミニOcamlの仕様を満たすために、次のような式の種類を$Syntax.ml$とし
て用いる。なお、これは講義ページで配布されたものをそのまま利用するため、
上記の仕様を含む、より多種の式の定義となっている。

\begin{lstlisting}
type exp = 
  | Var of string         (* variable e.g. x *)
  | IntLit of int         (* integer literal e.g. 17 *)
  | BoolLit of bool       (* boolean literal e.g. true *)
  | If of exp * exp * exp (* if e then e else e *)
  | Let of string * exp * exp   (* let x=e in e *)
  | LetRec of string * string * exp * exp   (* letrec f x=e in e *)
  | Fun of string * exp   (* fun x -> e *)
  | App of exp * exp      (* function application i.e. e e *)
  | Eq of exp * exp       (* e = e *)
  | Noteq of exp * exp (* e <> e *)
  | Greater of exp * exp  (* e > e *)
  | Less of exp * exp     (* e < e *)
  | Plus of exp * exp     (* e + e *)
  | Minus of exp * exp    (* e - e *)
  | Times of exp * exp    (* e * e *)
  | Div of exp * exp      (* e / e *)
  | Empty                 (* [ ] *)
  | Match of exp * ((exp * exp) list)    (* match e with e->e | ... *)
  | Cons of exp * exp     (* e :: e *)
  | Head of exp           (* List.hd e *)
  | Tail of exp           (* List.tl e *)
\end{lstlisting}

これらを踏まえて、与えられた式を評価して、最終的な値を計算するインタプリ
タを実装する。

\subsection{eval関数}

式を受け取り、式の種類によって違う処理を行うため、Ocamlのパターンマッチ
ングを用いて$eval$関数を実装する。整数値の足し算と掛け算を評価する$eval1$
を示す。

\begin{lstlisting}
let rec eval1 e =
  match e with
  | IntLit(n)    -> n
  | Plus(e1,e2)  -> (eval1 e1) + (eval1 e2)
  | Times(e1,e2) -> (eval1 e1) * (eval1 e2)
  | _ -> failwith "unknown expression"
\end{lstlisting}

もし与えられた式が{\bf{整数リテラル}}($IntLit$)だった場合は、その数値が式を評価
した値となる。また、与えられた式が足し算($Plus$)だった場合は、2つの引数
として与えられた式を再帰的に評価し、その2値を合計し、評価した値として返
す。掛け算についても同様である。\\
もしexp型の値であるが$IntLit,Plus,Times$でないものが渡された場合、最後の
パターン(\_)にマッチし、$unknown\ expression$という例外が投げられる。\\

この$eval1$に、入力として$2*2+(3+(-4))$という式を与えてみる。$exp$型の表現
では$Plus(Times(IntLit\ 2, IntLit\ 2), Plus(IntLit\ 3, IntLit\ (-4)))$である。
実行結果を以下に示す。

\begin{lstlisting}
val easy : exp =
  Plus (Times (IntLit 2, IntLit 2), Plus (IntLit 3, IntLit (-4)))
- : int = 3
\end{lstlisting}
正しく計算できていることが分かる。

\subsection{値と型}
$eval1$では評価した値はOcamlの$integer$型であるが、ミニOcamlのデータ型と
してはまず$Int$型と$Bool$型を扱いたいため、このままでは不十分である。複数のデー
タ型を扱えるようにするため、{\bf{値}}($value$)というデータ構造を導入する。

\begin{lstlisting}
type value = 
  | IntVal  of int        (* integer value e.g. 17 *)
  | BoolVal of bool       (* booleanvalue e.g. true *)
\end{lstlisting}

式を評価すると値が返る。整数型が求められる式上の位置に真偽値型がある場合
は型エラーであり、またその逆も同じくエラーである。例えば、条件節の真偽に
よって実行する処理を分岐するIf文では、条件節は真偽値型でなければならない。

以下に、$eval1$の式に加えて真偽値によって分岐する$If$、式1と式2が同じ値
なら真を返す$Eq$、式1が式2よりも大きい整数であれば真を返す$Greater$を実
装した$eval2$を以下に示す。


\begin{lstlisting}
let rec eval2 e =
  match e with
  | IntLit(n)  -> IntVal(n)
  | Plus(e1,e2) ->   
      begin
        match (eval2 e1, eval2 e2) with
	  | (IntVal(n1),IntVal(n2)) -> IntVal(n1+n2)
          | _ -> failwith "integer values expected"
      end
  | Times(e1,e2) ->
      begin
        match (eval2 e1, eval2 e2) with
	  | (IntVal(n1),IntVal(n2)) -> IntVal(n1*n2)
          | _ -> failwith "integer values expected"
      end
  | Eq(e1,e2) ->
      begin
	match (eval2 e1, eval2 e2) with
	  | (IntVal(n1),IntVal(n2)) -> BoolVal(n1=n2)
	  | (BoolVal(b1),BoolVal(b2)) -> BoolVal(b1=b2)
	  | _ -> failwith "wrong value"
      end
  | BoolLit(b) -> BoolVal(b)
  | If(e1,e2,e3) ->
      begin
	match (eval2 e1) with
	  | BoolVal(true) -> eval2 e2
	  | BoolVal(false) -> eval2 e3
	  | _ -> failwith "wrong value"
      end
  | Greater (e1, e2) ->
      begin
        match (eval2 e1, eval2 e2) with
          | (IntVal(n1), IntVal(n2)) -> BoolVal(n1 > n2)
          | _ -> failwith "wrong value"
      end
  | _ -> failwith "unknown expression e";;
\end{lstlisting}

このままでは$Plus$や$Times$の定義の大部分が重複しており、また後から二項
の整数演算を追加する際に煩雑になってしまう。これを回避するため、実際の計
算を$binop$関数に切り分けることにする。$binop$関数の定義は以下のとおり。

\begin{lstlisting}
  let binop f e1 e2 =
    match (eval2b e1, eval2b e2) with
    | (IntVal(n1),IntVal(n2)) -> IntVal(f n1 n2)
    | _ -> failwith "integer values expected"
\end{lstlisting}

これに伴って、$Plus,Times$の定義を以下のように変更する。

\begin{lstlisting}
...
    | Plus(e1,e2)  -> binop (+) e1 e2
    | Times(e1,e2) -> binop ( * ) e1 e2
...
\end{lstlisting}

\subsection{変数と環境}
ミニOcamlにおいて{\bf{変数}}を実現するため、$let$文の導入を試みる。変数を束ねて
おく{\bf{環境}}として、名前と値のタプルをリストにした$(string *
value)\ list$を考える。変数を宣言し
た際にはこの環境に名前と値を追加し、式の中に変数が出現した際にはこの環境
から名前を探しだして値を取り出すことが出来る。

\subsubsection{変数の定義・参照}
環境に変数を追加する操作として、環境の拡張$ext$という関数を実装する。こ
の関数は、引数に変数名と値、そして現在の環境を受け取り、先頭に追加したも
のを返す。

\begin{lstlisting}
let ext env x v = (x,v) :: env
\end{lstlisting}

また、現在の環境から変数を参照する関数として$lookup$を実装する。これは変
数名と現在の環境を受け取り、対応する値を返す。

\begin{lstlisting}
let rec lookup x env =
   match env with
   | [] -> failwith ("unbound variable: " ^ x)
   | (y,v)::tl -> if x=y then v 
                  else lookup x tl 
\end{lstlisting}

想定どおりの環境が作られているか、以下のようなテストプログラムを実行して
確認してみる。なお、$emptyenv$は空の環境を返す関数である。

\begin{lstlisting}
let env = emptyenv();;
let env = ext env "x" (IntVal 1);;
let env = ext env "y" (BoolVal true);;
let env = ext env "z" (IntVal 5);;

(* 実行結果 *)
val env : 'a list = []
val env : (string * value) list = [("x", IntVal 1)]
val env : (string * value) list = [("y", BoolVal true); ("x", IntVal 1)]
val env : (string * value) list =
  [("z", IntVal 5); ("y", BoolVal true); ("x", IntVal 1)]
\end{lstlisting}
正しく環境が作られていることが分かったので、変数名を参照してみる。

\begin{lstlisting}
let y = lookup "y" env;; (*存在する変数*)
let w = lookup "w" env;; (*存在しない変数*)

(*実行結果*)
val y : value = BoolVal true
Exception: Failure "unbound variable: w".
\end{lstlisting}
存在する変数は正しく参照できており、また未定義の変数は例外になっている。\\

なお、この環境は定義される度に先頭に追加されていく。$lookup$は先頭から順
に変数名の探索を行うため、同名の変数が存在する場合は最新の物が参照される
ため、変数名の上書きが可能になる。

\subsubsection{ミニOcamlへの組み込み}
実装した$ext,lookup$をミニOcamlインタプリタに組みこむことを考える。\\
まず、動作の理解を深めるため、環境の変化を順に表示するデバッグ関数
$string\_of\_env, print\_env$を実装した。

\begin{lstlisting}
let string_of_env env =
  let string_of_val (e:value) : string =
    match e with
      | IntVal n -> string_of_int n
      | BoolVal true ->  "true"
      | BoolVal false ->  "false"
  in
  let rec internal (env:(string * value) list) =
    match env with
      | [] -> ""
      | (name, v)::rest -> name ^ " = " ^ (string_of_val v) ^ ";" ^ (internal rest)
  in
  "[" ^ (internal env) ^ "]";;

let print_env (env: (string * value) list) =
  print_string( string_of_env env);;
\end{lstlisting}

これを踏まえて、変数を定義する式$Let$と変数を参照する式$Var$を追加した
$eval3$を以下に示す。デバッグモードを有効にするには、オプショナル引数と
して$\sim mode:1$を指定すると、環境の変化が逐一出力される。

\begin{lstlisting}
let rec eval3 ?(mode=0) e env =           (* env を引数に追加 *)
  if mode != 0 then print_env env;

  let binop f e1 e2 env =       (* binop の中でも eval3 を呼ぶので env を追加 *)
    match (eval3 e1 env ~mode, eval3 e2 env ~mode) with
    | (IntVal(n1),IntVal(n2)) -> IntVal(f n1 n2)
    | _ -> failwith "integer value expected"
  in 
  match e with
  | Var(x)       -> lookup x env
  | IntLit(n)    -> IntVal(n)
  | BoolLit(b) -> BoolVal(b)
  | Plus(e1,e2)  -> binop (+) e1 e2 env     (* env を追加 *)
  | Times(e1,e2) -> binop ( * ) e1 e2 env   (* env を追加 *)
  | Eq(e1,e2) ->
      begin
	match (eval3 e1 env ~mode, eval3 e2 env ~mode) with
	  | (IntVal(n1),IntVal(n2)) -> BoolVal(n1=n2)
	  | (BoolVal(b1),BoolVal(b2)) -> BoolVal(b1=b2)
	  | _ -> failwith "wrong value"
      end
  | If(e1,e2,e3) ->
      begin
        match (eval3 e1 env ~mode) with          (* env を追加 *)
          | BoolVal(true)  -> eval3 e2 env   (* env を追加 *)
          | BoolVal(false) -> eval3 e3 env   (* env を追加 *)
          | _ -> failwith "wrong value"
      end
  | Let(x,e1,e2) -> 
      let env1 = ext env x (eval3 e1 env ~mode)
      in eval3 e2 env1 ~mode
  | _ -> failwith "unknown expression";; 
\end{lstlisting}
テストとして、幾つかの入力を与えてみる。

\begin{lstlisting}
(* let x = 1 in 2 + x *)
eval3 (Let ("x", IntLit 1, (Plus (IntLit 2, Var "x")))) [];; 

(* let x = 1 in let y = 3 in x + y *)
eval3 (Let ("x", IntLit 1, Let("y", IntLit 3, (Plus (Var "y", Var "x"))))) [];;

(* let x = true in if x = true then 1 else 2 *)
eval3 (Let ("x", BoolLit true, If(Eq(Var "x", BoolLit true), IntLit 1, IntLit 2))) [];;

(* 実行結果 *)
- : value = IntVal 3
- : value = IntVal 4
- : value = IntVal 1
\end{lstlisting}

また、デバッグモードを用いて環境の遷移を表示してみる。

\begin{lstlisting}
(* let x = 1 in let y = x + 1 in x + y *)
eval3 ~mode:1 (Let ("x", IntLit 1, (Let ("y", Plus(Var "x", IntLit 1), Plus(Var "x", Var "y"))))) [];;

[][][x = 1;][x = 1;][x = 1;][x = 1;][y = 2;x = 1;][y = 2;x = 1;][y = 2;x = 1;]
- : value = IntVal 3
\end{lstlisting}
まず最初に$x$が定義され、次に$y$が定義されていく過程がわかる。

\section{構文解析}
今までは$exp$型の式を自分で作成していたが、$ocamllex$と$ocamlyacc$を用い
て、ミニOcaml言語の構文を解析して$exp$型の式を作り出すパーザを生成する。
parserとlexerは実験にあたって与えられた$parser.mly$と$lexer.mll$を用いる。
また、実行ファイルの生成にあたって必要となる他のファイル
($main.ml,syntax.ml$)も、与えられたものを用いることにする。

\subsection{構文の拡張}
lexerとparserを拡張して、新たに演算子を定義してみる。例として、
二数が等しくないと真を返す演算子$<>$を定義する。\\

$lexer.mll$に新しいトークンとして$<>$を$NOTEQUAL$という名前で追加する。

\begin{lstlisting}
 ...

  (* 演算子 *)
  | '+'       { PLUS }
  | '-'       { MINUS }
  | '*'       { ASTERISK }
  | '/'       { SLASH }
  | '='       { EQUAL }
  | '<'       { LESS }
  | '>'       { GREATER }
  | "::"      { COLCOL }
  | "<>"      { NOTEQUAL }

...
\end{lstlisting}
また、$parser.mly$を以下のように変更し、$NOTEQUAL$が出現した場合の処理として
$Noteq$という$exp$を出力するように定義する。
\begin{lstlisting}
...

// 演算子
%token PLUS     // '+'
%token MINUS    // '-'
%token ASTERISK // '*'
%token SLASH    // '/'
%token EQUAL    // '='
%token LESS     // '<'
%token GREATER  // '>'
%token COLCOL   // "::"
%token NOTEQUAL // "<>"

...

%left EQUAL GREATER LESS NOTEQUAL

...

  // e1 <> e2
  | exp NOTEQUAL exp
    { Noteq ($1, $3) }
    
...
\end{lstlisting}
最後に、$syntax.ml$に$Noteq$という$exp$を追加する。
\begin{lstlisting}
...

(* 式の型 *)
type exp = 
  | Var of string         (* variable e.g. x *)
  | IntLit of int         (* integer literal e.g. 17 *)
  | BoolLit of bool       (* boolean literal e.g. true *)
  | If of exp * exp * exp (* if e then e else e *)
  | Let of string * exp * exp   (* let x=e in e *)
  | LetRec of string * string * exp * exp   (* letrec f x=e in e *)
  | Fun of string * exp   (* fun x -> e *)
  | App of exp * exp      (* function application i.e. e e *)
  | Eq of exp * exp       (* e = e *)
  | Noteq of exp * exp (* e <> e *)
  | Greater of exp * exp  (* e > e *)
  | Less of exp * exp     (* e < e *)
  | Plus of exp * exp     (* e + e *)
  | Minus of exp * exp    (* e - e *)
  | Times of exp * exp    (* e * e *)
  | Div of exp * exp      (* e / e *)
  | Empty                 (* [ ] *)
  | Match of exp * ((exp * exp) list)    (* match e with e->e | ... *)
  | Cons of exp * exp     (* e :: e *)
  | Head of exp           (* List.hd e *)
  | Tail of exp           (* List.tl e *)

...
\end{lstlisting}

これらを踏まえて、$eval$に$Noteq$の処理を記述する。

\begin{lstlisting}
  | Noteq(e1, e2) ->
      begin
	match (eval3 e1 env ~mode, eval3 e2 env ~mode) with
	  | (IntVal(n1),IntVal(n2)) -> BoolVal(n1!=n2)
	  | (BoolVal(b1),BoolVal(b2)) -> BoolVal(b1!=b2)
	  | _ -> failwith "wrong value"
      end
\end{lstlisting}

以下のような入力を与えて動作を確認してみる。

\begin{lstlisting}
run "true <> false";;
run "(1+1) <> 2";;
run "let x = 1 in let y = 2 in x <> y";;

- : Syntax.value = Syntax.BoolVal true
- : Syntax.value = Syntax.BoolVal false
- : Syntax.value = Syntax.BoolVal true
\end{lstlisting}

正しく動いていることが分かる。

\subsection{関数と束縛}
ミニOcamlにおいて関数を実現するために、関数型という値を導入する。関数は
静的束縛を実現するため、名前と関数本体に加えて関数抽象が行われた時点での
環境を同時に保持する。\\

これに伴って、$value$型の定義を以下のように拡張する。
\begin{lstlisting}
type value = 
  | IntVal  of int        
  | BoolVal of bool       
  | FunVal  of string * exp * ((string * value) list) 
\end{lstlisting}

また、関数抽象を行う式$Fun$と関数適用を行う式$App$の処理を以下のように定
義する。

\begin{lstlisting}
...

  | Fun(x,e1) -> FunVal(x, e1, env)
  | App(e1,e2) ->
     begin
      match (eval4 e1 env) with
        | FunVal(x,body,env1) ->
            let arg = (eval4 e2 env) 
	    in eval4 body (ext env1 x arg)
        | _ -> failwith "function value expected"
     end

...
\end{lstlisting}

\subsection{再帰関数}
関数の再帰を実現するため、「再帰しない関数」とは別に再帰関数という型を用
意する。型の定義は以下のとおり。

\begin{lstlisting}
 | RecFunVal  of string * string * exp * ((string * value) list)
\end{lstlisting}

再帰関数を定義する式$LetRec$と、関数適用$App$に再帰関数を受け取った場合
の処理を追加し、$eval6$を作成する。これで、ミニOcamlインタプリタの基本的
な機能の実装は終了とする。\\

以下に、完成した$eval6$を含むミニOcamlインタプリタの全体を示す。

\begin{lstlisting}
open Syntax;;

let emptyenv () = [];;

let ext env x v = (x,v) :: env;;

let rec lookup x env =
   match env with
   | [] -> failwith ("unbound variable: " ^ x)
   | (y,v)::tl -> if x=y then v 
                  else lookup x tl;;

let string_of_env env =
  let string_of_val e =
    match e with
      | IntVal n -> string_of_int n
      | BoolVal true ->  "true"
      | BoolVal false ->  "false"
  in
  let rec internal env = 
    match env with
      | [] -> ""
      | (name, v)::rest -> name ^ " = " ^ (string_of_val v) ^ ";" ^ (internal rest)
  in
  "[" ^ (internal env) ^ "]";;

(* eval3 : exp -> (string * value) list -> value *)
(* let と変数、環境の導入 *)
let print_env env =
  print_string(string_of_env env);;

let rec eval6 e env =           (* env を引数に追加 *)
  let binop f e1 e2 env =       (* binop の中でも eval6 を呼ぶので env を追加 *)
    match (eval6 e1 env , eval6 e2 env ) with
    | (IntVal(n1),IntVal(n2)) -> IntVal(f n1 n2)
    | _ -> failwith "integer value expected"
  in 
  match e with
  | Var(x)       -> lookup x env
  | IntLit(n)    -> IntVal(n)
  | BoolLit(b) -> BoolVal(b)
  | Plus(e1,e2)  -> binop (+) e1 e2 env     (* env を追加 *)
  | Times(e1,e2) -> binop ( * ) e1 e2 env   (* env を追加 *)
  | Minus(e1, e2) -> binop (-) e1 e2 env
  | Eq(e1,e2) ->
      begin
	match (eval6 e1 env , eval6 e2 env ) with
	  | (IntVal(n1),IntVal(n2)) -> BoolVal(n1=n2)
	  | (BoolVal(b1),BoolVal(b2)) -> BoolVal(b1=b2)
	  | _ -> failwith "wrong value"
      end
  | If(e1,e2,e3) ->
      begin
        match (eval6 e1 env ) with          (* env を追加 *)
          | BoolVal(true)  -> eval6 e2 env   (* env を追加 *)
          | BoolVal(false) -> eval6 e3 env   (* env を追加 *)
          | _ -> failwith "wrong value"
      end
  | Let(x,e1,e2) -> 
      let env1 = ext env x (eval6 e1 env )
      in eval6 e2 env1 
 | LetRec(f,x,e1,e2) ->
      let env1 = ext env f (RecFunVal (f, x, e1, env))
      in eval6 e2 env1
  | Noteq(e1, e2) ->
      begin
	match (eval6 e1 env , eval6 e2 env ) with
	  | (IntVal(n1),IntVal(n2)) -> BoolVal(n1!=n2)
	  | (BoolVal(b1),BoolVal(b2)) -> BoolVal(b1!=b2)
	  | _ -> failwith "wrong value"
      end
  | Fun(x,e1) -> FunVal(x, e1, env)
  | App(e1,e2) ->
      let funpart = (eval6 e1 env) in
      let arg = (eval6 e2 env) in
        begin
         match funpart with
         | FunVal(x,body,env1) ->
            let env2 = (ext env1 x arg) in
            eval6 body env2
         | RecFunVal(f,x,body,env1) ->
            let env2 = (ext (ext env1 x arg) f funpart) in
            eval6 body env2
         | _ -> failwith "wrong value in App"
        end
  | Greater(e1, e2) -> 
      begin
        match (eval6 e1 env , eval6 e2 env ) with
          | (IntVal(n1), IntVal(n2)) -> BoolVal(n1 > n2)
          | _ -> failwith "wrong value"
      end
  | _ -> failwith "unknown expression";;
\end{lstlisting}


動作確認として、自然数$x$について階乗を求める再帰関数$fact$をミニOcamlで
実装し、実行した。結果は以下のとおり。

\begin{lstlisting}
let run src =
  Eval.eval6 (Main.parse(src)) (Eval.emptyenv())
;;

run "let rec fact x = if x = 0 then 1 else x * fact (x-1) in fact 10";;
\end{lstlisting}

\begin{lstlisting}
# val run : string -> Syntax.value = <fun>
- : Syntax.value = Syntax.IntVal 3628800
\end{lstlisting}
\section{ミニOcamlコンパイラの作成}
ここまで作成したミニOcamlインタプリタをもとに、仮想機械ASECの命令列を出
力するコンパイラを作成する。

\subsection{ASEC機械}
ASEC機械とは、$Accumulator, Stack, Environment, Code$の4部分で構成される
仮想機械である。資料として提示されたASEC仮想機械の遷移表と命令列の変換規
則にしたがって、今まで製作したインタプリタをもとにコンパイラを作成した。
\\

以下に、コンパイラとして製作した$eval$関数を含む$eval.ml$と、課題として
ASEC機械に機能追加を行った$am.ml$を示す。

\begin{lstlisting}[caption=eval.ml]
open Syntax;;
open Am;;

let emptyenv () = [];;

let rec position (x : string) (venv : string list) : int =
  match venv with
    | [] -> failwith "no matching variable in environment"
    | y::venv2 -> if x=y then 0 else (position x venv2) + 1;;

let rec eval e env =           (* env を引数に追加 *)
  let binop f e1 e2 env =       (* binop の中でも eval を呼ぶので env を追加 *)
    (eval e2 env) @ [I_Push] @ (eval e1 env) @ [f]
  in
  match e with
  | Var(x)       -> [I_Search (position x env)]
  | IntLit(n)    -> [I_Ldi n]
  | BoolLit(b) -> [I_Ldb b]
  | Plus(e1,e2)  -> binop I_Add e1 e2 env
  | Minus(e1,e2) -> binop I_Sub e1 e2 env
  | Times(e1,e2) -> binop I_Mult e1 e2 env
  | Div(e1,e2) -> binop I_Div e1 e2 env
  | Eq(e1,e2) -> binop I_Eq e1 e2 env
  | Noteq(e1,e2) -> binop I_Noteq e1 e2 env
  | If(e1,e2,e3) -> (eval e1 env) @ [I_Test ((eval e2 env), (eval e3 env))]
  | Let(x,e1,e2) -> 
      let env1 = x::env in
      [I_Pushenv] @ (eval e1 env) @ [I_Extend] @ (eval e2 env1) @ [I_Popenv]      
 | LetRec(f,x,e1,e2) -> 
     let env1 = f::env in
      [I_Pushenv] @ [I_Mkclos (eval e1 (x::env1))] @ [I_Extend] @ (eval e2 env1) @ [I_Popenv]
  | Fun(x,e1) -> [I_Mkclos (eval e1 (x::("_"::env)))]
  | App(e1,e2) -> [I_Pushenv] @ (eval e2 env) @ [I_Push] @ (eval e1 env) @ [I_Apply; I_Popenv] 
  | Greater(e1, e2) -> binop I_Greater e1 e2 env
  | _ -> failwith "unknown expression";;
\end{lstlisting}

\begin{lstlisting}[caption=am.ml(抜粋)]
...
  instr =    
  | I_Ldi of int            (* I_Ldi(n) は、整数nを Accumulator に置く (loadする) *)
  | I_Ldb of bool           (* I_Ldb(b) は、真理値bを Accumulator に置く (loadする) *)
  | I_Push                  (* Accumulatorにある値をスタックに積む *)
  | I_Extend                (* Accumulatorにある値を環境に積む (環境を拡張する) *)
  | I_Search of int         (* I_Search(i) は、環境の i 番目の値を取ってきて Accumulatorに置く *)
  | I_Pushenv               (* 現在の環境を、スタックに積む *)
  | I_Popenv                (* スタックのトップにある環境を、現在の環境とする *)
  | I_Mkclos of code        (* I_Mkclos(c) は、関数本体のコードが c で
			     * ある関数クロージャを生成し、Accumulatorに置く。
			     * なお、関数の引数は、変数名を除去しているので、
			     * このクロージャに含まれない。関数本体で
			     * は「環境の中の0番目の変数として指定される。
			     * ここでは、すべての関数を再帰関数として処理している。
			     *)
  | I_Apply                 (* Accumulator の値が関数クロージャである
			     * とき、その関数を、スタックトップにある
			     * 値に適用した計算を行なう。
			     *)
  | I_Test of code * code   (* I_Test(c1,c2)は、Accumulatorにある値が
			     * true ならば、コードc1 を実行し、false
			     * ならばコード c2 を実行する。
			     *)
  | I_Add                   (* Accumulatorにある値にスタックトップの値
			     * を加えて結果をAccumulator にいれる
			     *)
  | I_Sub                   (* Accumulatorにある値にスタックトップの値
                               を引いて結果をAccumulatorにいれる *)
  | I_Mult                  (* Accumulatorにある値にスタックトップの値
                               を掛けて結果をAccumulatorにいれる *)
  | I_Div                   (* Accumulatorにある値にスタックトップの値
                               で割って結果をAccumulatorにいれる *)
  | I_Greater               (* Accumulatorにある値がスタックトップの値
                               より大きいかどうかを比較して、結果をAccumulatorにいれる *)
  | I_Eq                    (* Accumulatorにある値とスタックトップの値
			     * が同じ整数であるかどうかをテストして、結
			     * 果をAccumulator にいれる
			     *)
  | I_Noteq                  (* Accumulatorにある値とスタックトップの値
			     * が違う整数であるかどうかをテストして、結
			     * 果をAccumulator にいれる
			     *)
...

(* ASEC machine の状態遷移(計算)*)
let rec trans (a:am_value) (s:am_stack) (e:am_env) (c:code) : am_value =
  let binop (a:am_value) (s:am_stack) (f:instr) : (am_value * am_stack) =
    begin
      match (f,      a,            s               ) with
	|   (I_Add,  AM_IntVal(n), AM_IntVal(m)::s1) -> (AM_IntVal (n+m), s1)
	|   (I_Sub,  AM_IntVal(n), AM_IntVal(m)::s1) -> (AM_IntVal (n-m), s1)
	|   (I_Mult,  AM_IntVal(n), AM_IntVal(m)::s1) -> (AM_IntVal (n*m), s1)
	|   (I_Div,  AM_IntVal(n), AM_IntVal(m)::s1) -> (AM_IntVal (n/m), s1)
	|   (I_Greater,  AM_IntVal(n), AM_IntVal(m)::s1) -> (AM_BoolVal (n>m), s1)
	|   (I_Eq ,  AM_IntVal(n), AM_IntVal(m)::s1) -> (AM_BoolVal(n=m), s1)
	|   (I_Noteq ,  AM_IntVal(n), AM_IntVal(m)::s1) -> (AM_BoolVal(n!=m), s1)
	| _ -> failwith "unexpected type of argument(s) for binary operation"
    end
  in 
  match (a,s,e,c) with 
    | (_,[],[],[]) -> a   (* コードが空の時、Accumulatorの値が最終結果 *)
    | (_,_,_,[]) -> failwith "non-empty stack or environment"  
	(* 計算が終わった時、スタックと環境は空であるべき。 *)
    | (_,_,_,I_Mkclos(c2)::c1) -> trans (AM_Closure(c2,e)) s e c1
    | (_,_,_,I_Push      ::c1) -> trans a (a::s) e c1
    | (_,_,_,I_Extend    ::c1) -> trans a s (a::e) c1
    | (_,_,_,I_Search(n) ::c1) -> trans (List.nth e n) s e c1
    | (_,_,_,I_Pushenv   ::c1) -> trans a (AM_Env(e)::s) e c1
    | (_,AM_Env(e1)::s1,_,I_Popenv::c1) -> trans a s1 e1 c1
    | (_,_::_          ,_,I_Popenv::c1) -> failwith "non-environment on the stack top"
    | (_,_             ,_,I_Popenv::c1) -> failwith "empty stack for Pop"
    | (AM_Closure(c2,e1),v::s1,_,I_Apply::c1) -> trans a s1 (v::a::e1) (c2@c1)
    | (_                ,_::_ ,_,I_Apply::c1) -> failwith "closure expected in accumulator for Apply"
    | (_                ,_    ,_,I_Apply::c1) -> failwith "empty stack for Apply"
    | (_,_,_,I_Ldi(n)::c1) -> trans (AM_IntVal(n))  s e c1
    | (_,_,_,I_Ldb(b)::c1) -> trans (AM_BoolVal(b)) s e c1
    | (AM_BoolVal(true), _,_,I_Test(c2,_ )::c1) -> trans a s e (c2 @ c1)
    | (AM_BoolVal(false),_,_,I_Test(_ ,c3)::c1) -> trans a s e (c3 @ c1)
    | (_,                _,_,I_Test(_ ,_ )::_ ) -> failwith "boolean expected for if-test"
    | (_,_,_,i       ::c1) 
	when List.mem i [I_Add; I_Sub; I_Mult; I_Div; I_Greater; I_Eq; I_Noteq] 
	-> 
	let (v,s1) = binop a s i
	in trans v s1 e c1
    | _ -> failwith "unknown instruction" 
...
\end{lstlisting}

\begin{lstlisting}
let run src =
  Eval.eval (Main.parse(src)) (Eval.emptyenv())
;;


let instr = run "let rec fact x = if x = 0 then 1 else x * fact (x-1) in fact 10";;
Am.am_eval instr;; 
\end{lstlisting}

\begin{lstlisting}
# val run : string -> Am.code = <fun>
val instr : Am.code =
  [Am.I_Pushenv;
   Am.I_Mkclos
    [Am.I_Ldi 0; Am.I_Push; Am.I_Search 0; Am.I_Eq;
     Am.I_Test ([Am.I_Ldi 1],
      [Am.I_Pushenv; Am.I_Ldi 1; Am.I_Push; Am.I_Search 0; Am.I_Sub;
       Am.I_Push; Am.I_Search 1; Am.I_Apply; Am.I_Popenv; Am.I_Push;
       Am.I_Search 0; Am.I_Mult])];
   Am.I_Extend; Am.I_Pushenv; Am.I_Ldi 10; Am.I_Push; Am.I_Search 0;
   Am.I_Apply; Am.I_Popenv; Am.I_Popenv]
- : Am.am_value = Am.AM_IntVal 3628800
\end{lstlisting}

正しく計算できていることが分かる。


\end{document}
