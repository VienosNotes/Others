\documentclass[a4j,9pt]{jsarticle}
\usepackage{stdrep}
\usepackage{ascmac}	% required for `\screen' (yatex added)
\usepackage{fancybox}

\title{認知科学概論 課題2}
\author{学籍番号 200911434 \\ 名前 青木大祐}
\begin{document}


\maketitle


\begin{screen}

Normanによる「ヒューマンエラーの分類」の3項目
(スリップ、ラプス、ミステーク)のそれぞれに該当す
る例を、身近な道具・機器・システムを使用中に経験
したものから各々1個あげて説明せよ。
\begin{itemize}
 \item 対象,使用状況,問題・失敗の流れを説明せよ。
 \item 各分類に該当する理由を説明せよ。
 \item 授業(他の授業も)にて取り上げたものは対象外。
 \item 先の宿題にて自身で報告したものでもよい。
\end{itemize}

\end{screen}
\vspace*{2zh}

\subsection*{スリップ}
MacOSにおいて、ブラウザなどのタブを閉じるcommand+Wとアプリケーションを終
了するcommand+Qを押し間違え、アプリケーションごと全てのタブが閉じられて
しまった。これはQWERTY配列のキーボードにおいてQとWが隣に配置されているために発生したミスであり、行為段
階での実行の誤りであるといえる。

\subsection*{ラプス}
操作中のコンピュータを遠隔で操作しているのを忘れており、ネットワークイン
ターフェースの設定中にifdownを実行してしまい、それ以上の遠隔操作が不可能
になってしまった。これは遠隔操作中であるという短期的な記憶が蒸発していた
ためだといえる。

\subsection*{ミステーク}
洗濯機の操作で、初めて予約を行おうとした際に「時間予約$\Rightarrow$開始」
を押すものだと思っていたが、実際は「時間予約」だけでよかった。開始を押す
と予約がキャンセルされ、即座に洗濯が実行されてしまう。これは行為の意図が
誤っていたため、想定していた結果とは違う状態になってしまったと言える。

\end{document}

