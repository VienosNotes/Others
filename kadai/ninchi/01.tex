\documentclass[a4j,9pt]{jsarticle}
\usepackage{stdrep}
\usepackage{ascmac}	% required for `\screen' (yatex added)
\usepackage{fancybox}

\title{認知科学概論 課題1}
\author{学籍番号 200911434 \\ 名前 青木大祐}
\begin{document}


\maketitle
\setcounter{subsection}{100}

\begin{screen}
身近な道具・機器・システムにおいて、使用中に経験した失敗(ヒューマンエラー)あるいは、気づいた問題。
人の認知的特性がその要因の1つになっている例を1個あげて説明せよ。
\begin{itemize}
 \item 対象,使用状況,問題・失敗の流れ,その結果を説明
 \item 授業(他の授業も)にて取り上げたものは対象外
 \item 独自の観点,一般に気づきにくい問題,より高く評価
\end{itemize}
\end{screen}
\vspace*{2zh}

運用が開始されたTwinsの新システムにおいて、「科目番号検索」に不可解な仕
様がある。ポップアップした科目番号検索ウィンドウのフォームに検索条件を入
力した上で検索を行うと、その条件にマッチした講義の一覧が表示されるが、親
ウィンドウで現在開いているモジュール(A、BまたはC)のペインと合致しない開講時期の講義が履修登
録できずエラーと表示される。例えば、「春学期A」のペインを開いた状態で、
検索条件「春学期」で検索結果として表示されるもののうち春学期Cに開講され
るものを登録しようとすると以下のようなエラーが表示される。
\vspace*{1zh}
\begin{lstlisting}
履修登録エラーです 内容を確認して下さい

該当モジュールでは履修登録できません。()
\end{lstlisting}
\vspace*{1zh}
これは人間の直感に反する挙動であり、仕様に問題があると考えられる。「従え
ない標識」の例と同様に、表示されている指示(を見て予想される結果)と、実
際に実行した時の結果が合致しない状態である。開講時期が一覧には表示されているが、通常の人間は検索結果に表示されている以上は
登録できると考えるのが一般的である。加えて言うならばTwinsの旧システムで
は可能な操作であった。間違ったモジュールの講義を登録しないようにという配慮であ
るなら、登録時に警告を出すなどの方法があったはずで、このような実体に則さ
ない上に、解決のための何の情報も得られないエラーメッセージを出すというデザインは「使いやすさ
の要因」の「エラー」において致命的な欠陥であると言える。



\end{document}

