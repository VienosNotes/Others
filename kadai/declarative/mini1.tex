\documentclass[a4paper,8pt]{jarticle}
\usepackage{url}
\usepackage[dvipdfmx]{graphicx}
\usepackage{epsfig}
\usepackage{amsmath}
\usepackage{amssymb}
\usepackage{times}
\usepackage{ascmac}
\usepackage{here}
\usepackage{txfonts}
\usepackage{listings}

\renewcommand{\lstlistingname}{リスト}
\lstset{%language=Ocaml,
  basicstyle=\ttfamily\scriptsize,
  commentstyle=\textit,
  classoffset=1,
  keywordstyle=\bfseries,
  frame=trbl,
  framesep=5pt,
  showstringspaces=false,
  numbers=left,
  stepnumber=1,
  numberstyle=\tiny,
  tabsize=2
}

 \lstdefinelanguage{CASL2}{
   morekeywords={C, C++, Ruby},
   morecomment=[l]{;},
   morestring=[b]",%"
 }

\setlength{\oddsidemargin}{-.2in}
\setlength{\evensidemargin}{-.2in}
\setlength{\topmargin}{-4em}
\setlength{\textwidth}{6.5in}
\setlength{\textheight}{9.5in}
\setlength{\parskip}{0em}
\setlength{\topsep}{0em}
\setlength{\columnsep}{3zw}

\title{宣言型プログラム論 ミニプロジェクト1}
\author{200911434 青木大祐}

\begin{document}
\maketitle
\newpage
以下にミニプロジェクト全体のソースコードを示す。
\lstinputlisting{mini.ml}
\newpage
\section{多倍長整数の加算}
\begin{quote}

\end{quote}
\lstinputlisting[firstnumber=23,linerange={23-39}]{mini.ml}
内部関数として、\tt{List.hd}、\tt{List.tl}に空リストの場合の処理を追加し
た\tt{myhd}、\tt{mytl}を用いており、各桁の和とcarryを計算しながら、再帰
的に次の桁へ計算を進めている。\\


以下に実行部分とその結果を示す。以降、テスト用の数値は別に作成した検算用プログラム
で生成したものを用いている。


\lstinputlisting[firstnumber=83,linerange={83-96}]{mini.ml}
\begin{lstlisting}
 - : int list = [458752885; 726968742; 1]
\end{lstlisting}
正しく計算できていることが分かる。

\section{n番目とn+1番目のフィボナッチ数の組を多倍長整数として計算する関
 数}
\lstinputlisting[firstnumber=41,linerange={41-48}]{mini.ml}
\tt{ni}を1から与えられた\tt{n}まで増やしながら、課題1で作成した
\tt{buint\_add}関数を用いて、フィボナッチ数を計算している。\\

\newpage
以下に実行部分とその結果を示す。\\

\lstinputlisting[firstnumber=98,linerange={98-106}]{mini.ml}
\begin{lstlisting}
- : int list * int list =
([1167376323; 1740041551; 76], [277887173; 604790509; 124]) 
\end{lstlisting}
正しく計算できていることが分かる。

\section{多倍長整数と整数 ($0$以上$2^{31}-1$以下) の乗算}
\lstinputlisting[firstnumber=50,linerange={50-60}]{mini.ml}
各桁に整数を掛けて下位からの繰り上がりを加算した後で、上位への繰り上がり
を算出し、再帰的に桁を辿っていく。\\

以下に実行部分とその結果を示す。
\lstinputlisting[firstnumber=108,linerange={108-114}]{mini.ml}
\begin{lstlisting}
- : int list = [444755861; 1144662764; 1592882151; 39]
\end{lstlisting}
正しく計算できていることが分かる。

\section{一桁分の整数 n に対して n! を多倍長整数として計算する関数}
\lstinputlisting[firstnumber=62,linerange={62-70}]{mini.ml}
課題3で作成した\tt{buint\_suint\_multi}関数を用いて、nowを0まで減らしなが
ら再帰的に乗算していく。\\

以下に実行部分とその結果を示す。
\lstinputlisting[firstnumber=116,linerange={116-121}]{mini.ml}
\begin{lstlisting}
- : int list = [2076180480; 1128227104; 3363457]
\end{lstlisting}
正しく計算できていることが分かる。

\section{多倍長整数の乗算}
\lstinputlisting[firstnumber=72,linerange={72-80}]{mini.ml}
同じく課題3で作成した\tt{buint\_suint\_multi}関数を用いて計算を行う。計
算を再帰に適した形にするため、今までは下位の桁から計算していたところを上
位の桁から計算している。\\

以下に実行部分とその結果を示す。
\lstinputlisting[firstnumber=123,linerange={123-128}]{mini.ml}
\begin{lstlisting}
- : int list = [1069672836; 1984622252; 393424931; 615602474]
\end{lstlisting}
正しく計算できていることが分かる。

\end{document}