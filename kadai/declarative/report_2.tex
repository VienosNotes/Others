\documentclass[a4paper,12pt]{jarticle}
\usepackage{url}
\usepackage[dvipdfmx]{graphicx}
\usepackage{epsfig}
\usepackage{amsmath}
\usepackage{amssymb}
\usepackage{times}
\usepackage{ascmac}
\usepackage{here}
\usepackage{txfonts}
\usepackage{listings}

\renewcommand{\lstlistingname}{リスト}
\lstset{%language=Ocaml,
  basicstyle=\ttfamily\scriptsize,
  commentstyle=\textit,
  classoffset=1,
  keywordstyle=\bfseries,
  frame=tRBl,
  framesep=5pt,
  showstringspaces=false,
  numbers=left,
  stepnumber=1,
  numberstyle=\tiny,
  tabsize=2
}

 \lstdefinelanguage{CASL2}{
   morekeywords={C, C++, Ruby},
   morecomment=[l]{;},
   morestring=[b]",%"
 }

\setlength{\oddsidemargin}{-.2in}
\setlength{\evensidemargin}{-.2in}
\setlength{\topmargin}{-4em}
\setlength{\textwidth}{6.5in}
\setlength{\textheight}{10in}
\setlength{\parskip}{0em}
\setlength{\topsep}{0em}
\setlength{\columnsep}{3zw}

\title{宣言型プログラム論}
\author{200911434 青木大祐}

\begin{document}
\maketitle
\newpage

\section*{問題3.1}
\lstinputlisting{./3_1.ml}
\begin{lstlisting}
 type figure = Circle of float | Square of float | Rectangle of float * float
 #             val area : figure -> float = <fun>
 #   12.56- : unit = ()
 # 4.- : unit = ()
 # 6.- : unit = ()
 # 
\end{lstlisting}

\section*{問題3.2}
\lstinputlisting{3_2.ml}
\begin{lstlisting}
type 'a tree = Lf | Br of 'a * 'a tree * 'a tree
#           val depth : 'a tree -> int = <fun>
#       val comptree : int -> 'a -> 'a tree = <fun>
#   val sample : int tree = Br (2, Br (4, Br (5, Lf, Lf), Lf), Br (1, Lf, Lf))
# - : int = 3
# - : string tree =
Br ("a",
 Br ("a",
  Br ("a", Br ("a", Br ("a", Lf, Lf), Br ("a", Lf, Lf)),
   Br ("a", Br ("a", Lf, Lf), Br ("a", Lf, Lf))),
  Br ("a", Br ("a", Br ("a", Lf, Lf), Br ("a", Lf, Lf)),
   Br ("a", Br ("a", Lf, Lf), Br ("a", Lf, Lf)))),
 Br ("a",
  Br ("a", Br ("a", Br ("a", Lf, Lf), Br ("a", Lf, Lf)),
   Br ("a", Br ("a", Lf, Lf), Br ("a", Lf, Lf))),
  Br ("a", Br ("a", Br ("a", Lf, Lf), Br ("a", Lf, Lf)),
   Br ("a", Br ("a", Lf, Lf), Br ("a", Lf, Lf)))))
\end{lstlisting}

\section*{問題3.3}
\lstinputlisting{3_3.ml}
\begin{lstlisting}
    type 'a tree = Lf | Br of 'a * 'a tree * 'a tree
#   val sample : int tree = Br (2, Br (4, Br (5, Lf, Lf), Lf), Br (1, Lf, Lf))
#         val inorder : 'a tree -> 'a list = <fun>
#         val postorder : 'a tree -> 'a list = <fun>
#   - : int list = [5; 4; 2; 1]
# - : int list = [5; 4; 1; 2]
\end{lstlisting}

\end{document}