\documentclass[a4paper,8pt,min]{jsarticle}
% \usepackage{url}
\usepackage[dvipdfmx]{graphicx}
% \usepackage{epsfig}
% \usepackage{amsmath}
% \usepackage{amssymb}
% \usepackage{times}
% \usepackage{ascmac}
\usepackage{txfonts}
%\usepackage{listings}


\setlength{\oddsidemargin}{-.2in}
\setlength{\evensidemargin}{-.2in}
\setlength{\topmargin}{-4em}
\setlength{\textwidth}{6.5in}
\setlength{\textheight}{10in}
\setlength{\parskip}{0em}
\setlength{\topsep}{0em}
\setlength{\columnsep}{3zw}

\title{TeXサーバの構築}
\author{200911434 青木大祐}

\begin{document}
\maketitle
\section{背景/目的}
講義のレポート課題や主専攻実験、プレゼンテーションスライドの作成などTeX
を用いる機会は多い。しかしTeXの導入はコストが高く、複数台の計算機を所有
した際に同様な環境を全てに整えるのは手間である。

これを解決するために、ネットワーク上に設置した1台のサーバにTeXファイル
を送信し、サーバでコンパイルしたPDFファイルをダウンロードする仕組みを構築
することを目的とする。

\section{実装目標}
\subsection{既存TeXファイルのコンパイル}
ローカル上にあるTeXファイルをサーバに送り、コンパイルさせる。

\subsection{WebUIでTeXファイルを記述}
普段使うTeXファイルのプリアンブルを予めテンプレートとして用意しておき、WebUIのフォームに
\textbackslash begin\{document\}以下を記述する。
WebUIを用いる利点としては、何も用意されていない他人の計算機をゲストユー
ザとして利用する場合や、OSをインストールしたばかりの計算機を使う際などの利便性が挙げられる。

\subsection{スクリプトによるTeXファイルの生成}
TeXよりも簡単なTrac Wiki記法を用いて構造的なテキストを記述し、サーバ側で
TeXに変換してコンパイルする。


また、進捗に応じて追加で必要と思われる機能を逐次実装していく。

\section{実装方針}
実装にはPerlのWeb Application FrameworkであるAmon2を利用する。このサービ
スは複雑なページ遷移やデータベースマネージャなどは必要無いため、軽量フレー
ムワークの方が記述量が少なく適していると考えられるためである。

\end{document}