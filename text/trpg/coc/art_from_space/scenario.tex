\documentclass[a4paper,8pt,min]{jsarticle}
\usepackage[dvipdfmx]{graphicx}
\usepackage{txfonts}
\usepackage{trpgmacro}
\usepackage{pxrubrica}
\usepackage{fancybox}

\title{名称未定 - シナリオ案}
\author{VienosNotes}

\begin{document}
\maketitle
\newpage
\section{シナリオ概要}
二学期の試験も終わりに近づき、そろそろ秋休みの雰囲気が漂ってきたある日の
こと。オカ研の顧問である都筑は、彼の旧友であるという一人
の女性を部員の前に連れてきた。\\


「ある人------画家を探して欲しいんです」\\


その女性---三船早紀はある画家のマネージャをしており、雇い主である画家の
「外山昭三」を探して欲しいという。話を聞くには、どうやら外山はかなりの変
わり者らしく、すべての打ち合わせはメールで行い、人には会おうとしないらし
い。その外山であるが、ある時を境にパッタリと連絡が途絶えてしまい、完全に
消息不明になってしまっているようだ。

プレイヤーたちの目標は、数少ない情報から外山の消息をたどること。外山は無
事なのだろうか?無事でないならば、いったい外山に何があったのか?
果たしてオカ研の部員たちは、無事に真相にたどり着くことができるのだろうか。


% * 画家の名前は外山昭三(とやましょうぞう)
% * どこに住んでいるのかは分からないが、つい最近までは手紙と絵を送ってきていた
% * 数ヶ月前から音信不通になっている
% * 手がかりは送られてきていた手紙の消印と、同封されていた写真など


% 外山は人嫌いで有名で、売れる絵を描いているが極力人に会おうとはせず、ほとんどの商談は手紙で行っている。生活に必要なものは自分で調達しているようであり、業界人以外の付近住民とは普通に接していたようである。

% 調べてみると、外山の画風は一年前を境に豹変している。今までは精密なタッチで描かれた風景画を専門としていたが、何かが起きてからはサイケデリックな色彩の荒いタッチになっている。

% ちょうど外山の画風が変化した頃に、その付近でUFO騒ぎがあった。公式には雷か何かだと発表されているが、直接黙視したものの証言では雷の色ではなかったと言う。

\section{設定}
\subsection{時期}
2011年11月25日から12月1日

\subsection{舞台}
\subsubsection{筑波大学}
国立大学法人筑波大学。シナリオ「星を見る少女」と共通。

\subsubsection{外山邸}
高知県\ruby{幡多}{は|た}郡大月町。窓からは風車が並ぶ尾根を一望でき、海も
近い。
\subsection{登場人物}
\subsubsection{\ruby{外山}{と|やま}\ruby{昭三}{しょう|ぞう}}
43歳、男性。画家。人付き合いが嫌いで、人里離れた山奥で独り絵を描く。現在
は消息不明で、オカ研に捜索依頼が出される。

\subsubsection{\ruby{都筑}{つ|づき}\ruby{稔}{みのる}} 
51歳、男性。筑波大学准教授。オカルト研究会こと「都市文化研究会」の顧問を務める。

\subsubsection{\ruby{三船早希}{み|ふね|さ|き}}
28歳、女性。画廊「Light Factory」勤務。都筑に連れられてオカ研を訪れる。かつての都筑の教え
子。消息不明になった担当の画家の行方を探るため、都筑を頼る。

\newpage
\section{シナリオ本編}
\subsection{導入 --- オカ研}
2011年11月25日。


二学期の期末試験も終わり、部室で脱力している部員たち。今年は2日ほど例年
より秋休みが長いので、「どこか旅行でも行こうか」という話をしている。\\

(適当にロールプレイ)\\


鐘が鳴る。時間は18時を回ったところ。講義が終わった都筑が戻ってきた。都筑
のあとから、「失礼します」と言って見知らぬ女性が入ってくる。都筑は部屋を
見渡してメンバーが揃っていることを確認すると、「ちょっと皆、いいかな。集
まって欲しいんだけど」といってメンバーを応接室のソファに集め、話し始めた。

「こちらは三船早希くん。何年か前の僕の教え子だ。」
都筑はそう言って女性を紹介した。\\

「彼女は都内の画廊に務めているんだが、ちょっと問題が発生したようでね。そ
れで僕を頼って連絡をくれたわけなんだけど、せっかくなんで君等にも手伝っ
てもらおうかなと思ったわけさ------とりあえずコーヒーでも淹れようか。その間
に君らは彼女から詳しい話を聞いてみてくれ」


そう言って都筑は給湯室に向かった。都筑は「何か話をするならコーヒーがないと
話にならない」と豪語するほどのカフェイン中毒者である。\\


紹介された女性は「はじめまして、三船です」と言って名刺を配った。黒のビジネススー
ツに黒縁のメガネというシックな服装で、いかにも働く女性といった装いである。


「相談というのも単刀直入に申しますと、ある人------画家を探して欲しいんで
す。」
と早希は話し始めた。

「私は都内にあります『Light Factory』という画廊で働いており、主に画
家の先生とご商談の契約や折衝などを務めています。
少し前のことです。私が担当しております『外山昭三』という先生に、ふた月程前
から突然連絡がつかなくなってしまったのです。外山先生は大層な人間嫌いで、
私との商談もメールだけで済ませてしまわれる方でしたが、同時に几帳面な方で
もありました。今まで何度もメールでやり取りをしてきましたが、3日も返信が
遅れたことはありませんでした」

そういって早希はうつむいて唇を噛んだ。
「私も外山先生専属というわけでもありませんし、こういった人探しというのも
経験がありませんので、探偵業の方でも紹介して頂こうと都筑先生を頼った次
第です」\\

(質問タイム)


\begin{topic}
 \item[きっかけに思い当たることはないか?]
 「直前までのメールも今までどおりでしたし、特に思い当たることは…」と言っ
 て少し考え込んだあと、「そういえば」と切り出す。「ちょうど一年ほど前に、
 先生はがらりと作風を変えられました。今までは繊細なタッチで穏やかな風景
 画を描かれていたのですが、あの頃からは色合いが独特な絵を描かれるように
 なりました。風景を描かれているのは同じなのですが、画風を変えたあとのほ
 うが人気が出ているようです」

\item[なぜ警察に頼まないのか?]
 「実のところ、私達も先生の正確な居場所を把握しておりませんので、警察の
 \ruby{方}{かた}も『これだけの情報では警察が動くのは難しい』と。」

\item[居場所に関して手がかりはないのか?]
 「先生は作品をご自身の車で運搬業者に持ち込まれているようで、あまり詳し
 いご住所はわかっておりません。持ち込み先は四国の業者が多いようですので、
 あのあたりにお住まいである可能性は高いと思います」\\
 
\end{topic}
\newpage

そうして話をしていると、都筑がお盆にコーヒーを載せて戻ってきた。「どうだ
い、大体の話は掴めたかい?」とPCたちに問いかける。\\


(適当にロールプレイ)\\

「前回の事件で、思ってたより君らが優秀だったのがわかったからね。僕は他な
らぬ三船くんの頼みだから手伝うのは当然なんだけど、もし良かったら君らにも
手伝ってもらいたいと思う。もちろん報酬は出すよ」\\
そう言って都筑はニヤリと笑った。
「君らには、この秋休みを利用して外山画伯を探しに現地へと向かってもらいた
い。報酬として、掛かった交通費、旅費は全て僕が持とう。ちょっとしたオカ研
の合宿というわけだ------なかなか悪い話でもないだろう?」\\

この一見巫山戯たようにも見える都筑の提案だが、以前は都筑に付いて研究して
いたと言うだけあって早希は真剣な表情をしている。
「もし受けて頂けるのであれば、私からも何かのお礼を差し上げたいと思います。
どうか、先生の安否を確認して頂けないでしょうか」\\

(適当にPLを言いくるめて依頼を受けてもらう)\\

「ありがとうございます。何かの手がかりになるかもしれませんので、先生の画
集と、未発表の作品のコピーをお預けしておきます。何か不明な点など有りまし
たら先ほどの名刺にあります電話番号までご連絡をお願いいたします」
t
そう言って早希は画廊へと戻っていった。

\subsection{調査開始 --- オカ研}
\subsubsection{外山昭三についての調査}

\begin{judge}{図書館}
 \item 以下の情報が得られる。
 \item 特になし。
\end{judge}

\begin{topic}
  \item[本人について]
 東京芸術大学卒、43歳。資産家の家に生まれ、幼少の頃から不自由なく生活し
 ている。彼が12歳の頃に両親が他界し、相当額の遺産を受け継いだ。\\
 幼い頃から目を患っており、後天的に色覚異常を持っていたが手術により完治。
 初めて見る本当の世界の美しさに心を打たれ、美術を志すようになった。その
 後は順当に大学を卒業し、今から20年ほど前に画家として活動を開始したよう
 だ。また、極度の人間嫌いという点で業界でも有名である。

  \item[描く絵について]
  写実的な風景画を得意としていたようだ。精密な書き込みと素朴な色使いで、
  少ないながらも固定ファンのついている画家だったようだ。\\
  一年ほど前に出した新作「風流るる色」では今までの画風を脱ぎ捨てた大胆な色
  使いの新境地に達し、新たなファンを獲得した。
\end{topic}

\subsubsection{渡された画集について}


\end{document}

