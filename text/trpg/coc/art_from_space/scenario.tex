\documentclass[a4paper,8pt,min]{jsarticle}
\usepackage[dvipdfmx]{graphicx}
\usepackage{txfonts}
\usepackage{trpgmacro}
\usepackage{pxrubrica}
\usepackage{fancybox}

\title{{\large{クトゥルフ神話TRPGシナリオ}}\\ 懐かしい色}
\author{VienosNotes}

\usepackage{ascmac}	% required for `\itembox' (yatex added)
\begin{document}
\maketitle
\thispagestyle{empty}
\newpage
\setcounter{page}{1}
\section{シナリオ概要}
二学期の試験も終わりに近づき、そろそろ秋休みの雰囲気が漂ってきたある日の
こと。オカ研の顧問である都筑は、彼の教え子であるという一人
の女性を部員の前に連れてきた。\\


「ある人------画家を探して欲しいんです」\\


その女性---三船早紀はある画家のマネージャをしており、雇い主である画家の
「外山昭三」を探して欲しいという。話を聞くには、どうやら外山はかなりの変
わり者らしく、すべての打ち合わせはメールで行い、人には会おうとしないらし
い。その外山であるが、ある時を境にパッタリと連絡が途絶えてしまい、完全に
消息不明になってしまっているようだ。

プレイヤーたちの目標は、数少ない情報から外山の消息をたどること。外山は無
事なのだろうか?無事でないならば、いったい外山に何があったのか?
果たしてオカ研の部員たちは、無事に真相にたどり着くことができるのだろうか。


% * 画家の名前は外山昭三(とやましょうぞう)
% * どこに住んでいるのかは分からないが、つい最近までは手紙と絵を送ってきていた
% * 数ヶ月前から音信不通になっている
% * 手がかりは送られてきていた手紙の消印と、同封されていた写真など

% 外山は人嫌いで有名で、売れる絵を描いているが極力人に会おうとはせず、ほ
% とんどの商談は手紙で行っている。

% 同じ山並みを描いた2つの絵から風車が建築された時期が判明し、大月ウィン
% ドファームという風力発電所の近くであることを突き止める

% 調べてみると、外山の画風は一年前を境に豹変している。今までは精密なタッ
% チで描かれた風景画を専門としていたが、何かが起きてからはサイケデリック
% な色彩の荒いタッチになっている。

% ちょうど外山の画風が変化した頃に、その付近でUFO騒ぎがあった。公式には
% 雷か何かだと発表されているが、直接黙視したものの証言では雷の色ではなかっ
% たと言う。

% 実際にはそれは"宇宙からの色"の飛来であり、その色彩に外山はかつて自分が
% 見ていた世界を思い出した。

% 外山はもともと色覚異常を持っていた。"宇宙からの色"に触れ、当時抱いていた
% 漠然とした恐怖を思い出した。しかしその恐怖は外山にとって過去の自分その
% ものであり、自らの原点であるノスタルジーを求めて絵にすることにした。

% またそれだけでは飽きたらず、自宅のプールに"宇宙からの色"が潜んでいるこ
% とを知り、その色彩を求めてプールに飛び込み、生命力を吸われて即死した。

 \section{設定}
\subsection{時期}
2011年11月25日から12月1日

\subsection{舞台}
\subsubsection{筑波大学}
国立大学法人筑波大学。シナリオ「星を見る少女」と共通。

\subsubsection{外山邸}
高知県\ruby{幡多}{は|た}郡大月町。窓からは風車が並ぶ尾根を一望でき、海も
近い。東京から飛行機と鉄道を使って7時間くらいかかる。

\subsection{登場人物}
\subsubsection{\ruby{外山}{と|やま}\ruby{昭三}{しょう|ぞう}}
43歳、男性。画家。人付き合いが嫌いで、人里離れた山奥で独り絵を描く。現在
は消息不明で、オカ研に捜索依頼が出される。

\subsubsection{\ruby{都筑}{つ|づき}\ruby{稔}{みのる}} 
51歳、男性。筑波大学准教授。オカルト研究会こと「都市文化研究会」の顧問を務める。

\subsubsection{\ruby{三船早希}{み|ふね|さ|き}}
28歳、女性。画廊「Light Factory」勤務。都筑に連れられてオカ研を訪れる。かつての都筑の教え
子。消息不明になった担当の画家の行方を探るため、都筑を頼る。

\newpage
\section{シナリオ本編}
\subsection{導入 --- オカ研}
2011年11月25日。


二学期の期末試験も終わり、部室で脱力している部員たち。今年は2日ほど例年
より秋休みが長いので、「どこか旅行でも行こうか」という話をしている。\\

(適当にロールプレイ)\\


鐘が鳴る。時間は18時を回ったところ。講義が終わった都筑が戻ってきた。都筑
のあとから、「失礼します」と言って見知らぬ女性が入ってくる。都筑は部屋を
見渡してメンバーが揃っていることを確認すると、「ちょっと皆、いいかな。集
まって欲しいんだけど」といってメンバーを応接室のソファに集め、話し始めた。

「こちらは三船早希くん。何年か前の僕の教え子だ」
都筑はそう言って女性を紹介した。\\

「彼女は都内の画廊に務めているんだが、ちょっと問題が発生したようでね。そ
れで僕を頼って連絡をくれたわけなんだけど、せっかくなんで君等にも手伝っ
てもらおうかなと思ったわけさ------とりあえずコーヒーでも淹れようか。その間
に君らは彼女から詳しい話を聞いてみてくれ」


そう言って都筑は給湯室に向かった\footnote{都筑は「何か話をするならコーヒーがないと
話にならない」と豪語するほどのカフェイン中毒者である。}。\\


紹介された女性は「はじめまして、三船です」と言って名刺を配った。黒のビジネススー
ツに黒縁のメガネというシックな服装で、いかにも働く女性といった装いである。


「相談というのも単刀直入に申しますと、ある人------画家を探して欲しいんで
す。」
と早希は話し始めた。

「私は都内にあります『Light Factory』という画廊で働いており、主に画
家の先生とご商談の契約や折衝などを務めています。
少し前のことです。私が担当しております『外山昭三』という先生に、ふた月程前
から突然連絡がつかなくなってしまったのです。外山先生は大層な人間嫌いで、
私との商談もメールだけで済ませてしまわれる方でしたが、同時に几帳面な方で
もありました。今まで何度もメールでやり取りをしてきましたが、3日も返信が
遅れたことはありませんでした」

そういって早希はうつむいて唇を噛んだ。
「私も外山先生専属というわけでもありませんし、こういった人探しというのも
経験がありませんので、探偵業の方でも紹介して頂こうと都筑先生を頼った次
第です」\\

(質問タイム)


\begin{topic}
 \item[きっかけに思い当たることはないか?]
 「直前までのメールも今までどおりでしたし、特に思い当たることは…」と言っ
 て少し考え込んだあと、「そういえば」と切り出す。「ちょうど一年ほど前に、
 先生はがらりと作風を変えられました。今までは繊細なタッチで穏やかな風景
 画を描かれていたのですが、あの頃からは色合いが独特な絵を描かれるように
 なりました。風景を描かれているのは同じなのですが、画風を変えたあとのほ
 うが人気が出ているようです」

\item[なぜ警察に頼まないのか?]
 「実のところ、私達も先生の正確な居場所を把握しておりませんので、警察の
 \ruby{方}{かた}も『これだけの情報では警察が動くのは難しい』と。」

\item[居場所に関して手がかりはないのか?]
 「先生は作品をご自身の車で運搬業者に持ち込まれているようで、あまり詳し
 いご住所はわかっておりません。持ち込み先は四国の業者が多いようですので、
 あのあたりにお住まいである可能性は高いと思います」\\
 
\end{topic}
\newpage

そうして話をしていると、都筑がお盆にコーヒーを載せて戻ってきた。「どうだ
い、大体の話は掴めたかい?」とPCたちに問いかける。\\


(適当にロールプレイ)\\

「前回の事件\footnote{シナリオ:星を見る少女}で、思ってたより君らが優秀だったのがわかったからね。僕は他な
らぬ三船くんの頼みだから手伝うのは当然なんだけど、もし良かったら君らにも
手伝ってもらいたいと思う。もちろん報酬は出すよ」\\
そう言って都筑はニヤリと笑った。
「君らには、この秋休みを利用して外山画伯を探しに現地へと向かってもらいた
い。報酬として、掛かった交通費、旅費は全て僕が持とう。ちょっとしたオカ研
の合宿というわけだ------なかなか悪い話でもないだろう?」\\

この一見巫山戯たようにも見える都筑の提案だが、以前は都筑に付いて研究して
いたと言うだけあって早希は真剣な表情をしている。
「もし受けて頂けるのであれば、私からも何かのお礼を差し上げたいと思います。
どうか、先生の安否を確認して頂けないでしょうか」\\

(適当にPLを言いくるめて依頼を受けてもらう)\\

「ありがとうございます。何かの手がかりになるかもしれませんので、先生の画
集と、未発表の作品のコピーをお預けしておきます。何か不明な点など有りまし
たら先ほどの名刺にあります電話番号までご連絡をお願いいたします」

そう言って早希は画廊へと戻っていった。

\subsection{調査開始 --- オカ研}
\subsubsection{外山昭三についての調査}

\begin{judge}{図書館}
 \item 調べた内容に最も近いトピックを得る。

\end{judge}

\begin{topic}
  \item[本人について]
 東京芸術大学卒、43歳。資産家の家に生まれ、幼少の頃から不自由なく生活し
 ている。彼が12歳の頃に両親が他界し、相当額の遺産を受け継いだ。\\
 幼い頃から目を患っており、後天的に色覚異常を持っていたが手術により完治。
 初めて見る本当の世界の美しさに心を打たれ、美術を志すようになった。その
 後は順当に大学を卒業し、今から20年ほど前に画家として活動を開始したよう
 だ。また、極度の人間嫌い\footnote{どうやら幼少のころの病気の経験が影響しているようだ。}という点で業界でも有名である。

  \item[描く絵について]
  写実的な風景画を得意としており、精密な書き込みと素朴な色使いで、
  ある程度の固定ファンがついている画家だったようだ。\\
  一年ほど前に出した新作「風流るる色」では今までの画風を脱ぎ捨てた大胆な色
  使いの新境地に達し、新たなファンを獲得した。
\end{topic}
\newpage

\subsubsection{渡された画集について}
「外山昭三全集」と題された画集と、何枚かのA4紙がホチキスで止められた束が
ある。前者はおおよそ2006年までに発表された作品が掲載されており、紙束のほ
うはそれより後に発表された作品のようだ。各作品にはタイトルと発表年が記載
されている。
どの絵も風景画であり、山や海など自然の風景が多い。特徴的な地形はあまりな
く、単純に絵に書かれている地形のみから題材を割り出すのは難しいだろう。何
か目立つ建造物などが描かれていないだろうか…。\\

また、画風が変わってからの絵はサイケデリックな色合いになり、一見風景画と
もわからないような絵になっている。\\

全集には、外山が書いた前書きが掲載されている。

\begin{itembox}{外山昭三全集 - 前書きより抜粋}
 (前略)私はいわゆる名所へ足を運び、絵を描くということはしないようにして
 いる。美しいと言われる場所へ行って美しいと感じるのは、何か足りない気が
 してならないのだ。所詮、そこを誰かが美しいと言った時点で何かしらのバイ
 アスに囚われてしまう。\\
 私が描きたいほんとうの美しさというものは、私達が幾度と無く見てきた、生活に
 密着した景色のなかにある日見つけた驚きや喜び、そういった物の中にこそ潜
 んでいる。
\end{itembox}

\begin{judge}{目星}
 \item 以下の全トピックを得る。
\end{judge}

\begin{topic}
 \item[描かれている風景について]
 「普段見ている風景を描く」というだけあって、派手ではないが心に染み入るよう
 な美しさが素朴な色使いで描かれている。季節を変えて同じ場所を描いた作品
 も多く、またその中には自宅の窓から見える風景を描いたような作品もある。 

 \item[外山邸?]
 寝ている犬を描いた「午睡」に、外山氏の自宅と思われる小さな洋館が描かれ
 ている。建物は小ぶりだが立派な庭があり、プールがあるのが見える。周りに
 人家はなく、林が広がっているようだ。

\end{topic}

\begin{judge}{アイデア[地質学に成功すれば+10]}
 \item トピック[風車について][描かれた時期]を得る。
 \item トピック[風車が見える]を得る。
\end{judge}

\begin{topic}
 \item[風車について]
 おそらく同じ山を描いたと思しき二枚の作品がある。片方は夕暮れの霧に沈む山並みを
 描いた作品であるが、もう一枚の爽やかな朝日が山から昇るさまを窓から望む
 絵には前者にはなかった風車が描かれていることに気づく。


 \item[描かれた時期]
 まだ風車が描かれていないほうの絵「雲海に沈む」は2005年の7月に、風車が描
 かれた「緑風」は2006年の12月に描かれたようだ。

 \item[風車が見える]
 山並みに風車が点在している絵「緑風」があることに気づく。
\end{topic}
\newpage
\subsubsection{風力発電所についての調査}
トピック[描かれた時期]を得ているなら、風力発電所が作られた時期から場所の
割り出しを試みることができる。また、[風車が見える]を得ているなら、困難で
はあるが風車の数や規模から発電所の特定を試みることができる。

\begin{judge}{図書館}
 \item トピック[四国の風力発電所][その時期に完成したもの]を得る。また、
 ロールプレイで要求されれば[どちらの風力発電所か]を追加で得られる。
 \item トピック[四国の風力発電所]を得る。
\end{judge}

\begin{topic}
 \item[四国の風力発電所]
 四国では風力発電がそれなりに盛んであり、15箇所ほどの風力発電所が稼働中
 である。
 \item[その時期に完成したもの]
 2つの絵が描かれた間に完成した風力発電所は2つある。ひとつは高知県にある
 「大月ウィンドファーム」であり、もう一つは同じく高知県にある「葉山風力
 発電所」である。

 \item[どちらの風力発電所か]
 「緑風」には12機の風車が描かれている。「大月ウィンドファーム」は絵と一
 致する12機で運用されており、「葉山風力発電所」は20機で運用されている。
 このことから、おそらくは前者が描かれているものと推測される。
\end{topic}

発電所が特定できれば、窓から見た位置関係と朝日の方角から、大体の位置関係
を割り出すことができるだろう。また、住所が割り出せればオンラインで登記情
報を確認することができるかもしれない。

\subsubsection{画風の豹変について}
外山のおおまかな居場所を特定した時点で、外山の画風が変化した理由について
の調査ができるようになる。\\

画風が一変した頃に、大月町付近でなにか変わった出来事がなかったかを調べる。

\begin{judge}{図書館}
 \item トピック[UFO騒ぎ]を得る。
\end{judge}

\begin{topic}
 \item[UFO騒ぎ]
 大月町に住んでいる高校生のブログで、一件だけ気になるものが。\\

 「西の空に妙な光が飛んでいるのが見えた後、大きな音がしたと思ったら空が強く光って元に
 戻った。直接見ていなかった親は雷か何かじゃないかと言っているけれども、
 あれは雷なんかじゃなかったと思う。

 あんな色の空は見たことがなかったし、光もまっすぐに落ちていった。雷なら
 ギザギザの光が見えるはず。同じく目撃した友人とUFOかもしれないと話をした」
\end{topic}
\newpage

\subsection{現地調査}
外山の自宅を突き止めたPCたちは、8時間かけて高知県大月町を訪れた。宿泊先
は都筑が手配した旅館「中田旅館」である。町の人々への聞き込みで以下の様な情
報が得られる。

\subsubsection{大月町}

\begin{topic}
 \item[洋館について]
 確かに街のはずれに古い洋館があり、画家の先生が住んでいるという話が聞け
 る。洋館は海岸側から少し山に入ったあたりに建っているが、山には風車以外
 は特になにもないため、業者以外は山に入ることは少ないという。
 \item[UFOについて]
 さすがに半年も前のことなので、覚えている人はいない。
 \item[何か変わったことはないか]
 半年前くらいからカラスがやたら増えて、ゴミが荒らされて困る。
\end{topic}

\subsubsection{外山邸}

最寄りのバス停から歩いて洋館へ向かう。舗装はされていないが、車が通れるく
らいの山道が続いている。
\begin{judge}{ナビゲート}
 \item 一時間ほどで洋館に着く
 \item 道に迷って三時間かかる
\end{judge}

洋館にたどり着いたが、人の気配はしない。庭も手入れされている様子はなく荒
れ放題である。門には鍵が掛かっているが、乗り越えられないこともない。\\

以下のポイントは特に判定無く捜索できる。
\begin{topic}
 \item[庭]
 庭の方へと行ってみると、絵にあった通りの犬小屋やプールがある。プールは
 長らく使われた様子はなく、水はドドメ色に濁っている。犬小屋に犬はいない。

 \item[家の周り]
 ぐるりと家の周りを回ってみると、開け放たれた窓がある。窓の\ruby{桟}{さ
 ん}には汚れが溜まっており、長い間開け放たれていたことが伺える。

 \item[一階]
 家の中にも埃が溜まっている。古い家具や額縁などが並んでおり、年季を感じ
 させる内装になっている。
\end{topic}

一階を探索していると、外山が書斎として使っていたと思われる部屋が見つかる。
\begin{judge}{目星}
 \item トピック[外山の日記1]を得る。
\end{judge}

\begin{topic}
 \item[外山の日記1] 
 2010年の3月31日までの日記が本棚に収まっている。「庭の花が咲いた」や
 「\ruby[g]{紅葉}{もみじ}の色が変わってきた」など、画家らしく絵のモチー
 フについての記述が多い。\\
 最新のものであるはずの2011年の日記は見当たらないが、おそらくもともと入っ
 ていたであろう場所に隙間がある。
\end{topic}
\begin{itembox}{外山の日記1}
 
 \tt{2010年 11月11日}
 
 今日は朝から空模様が怪しかったが、昼過ぎになって雨がふりだした。\\
 夕方の五時頃だっただろうか、庭ですごい音がしたので振り向くと、強烈な光
 に目を焼かれた。今まで見たことのないような色の光だったが、何故か懐かし
 い感じがした。いったい何だったのだろうか。\\
 プールの水面がひどく荒れていたので、隕石か何かが落ちてきたのかもしれな
 い。\\

 \tt{2010年 11月12日}

 今日は一日、あの光について考えていたが、ようやくあの懐かしい感覚の正体
 を掴んだと思う。\\
 私はかつて、あのような色だけを見て過ごしていたのだ。完全に同じ色であっ
 たかどうかは定かではないが、近いものがあったのだと思う。\\

 \tt{2010年 11月13日}

 どうもあの光を見てからというもの、何か落ち着かないまま時間だけが過ぎて
 いく。どうしても、かつて私が見ていた世界が思い起こされるのだ。今描いて
 いる絵も捗らない。\\
 ケリーの様子がおかしいのも気にかかる。いつもはもっと落ち着いているのだ
 が、昨日からずっとソワソワしているようだし、急に焦ったように吠え始める
 のも心配だ。今までこんなことはなかった。\\

 \tt{2010年 11月14日}

 やはりこのままではいけない。画家たるもの、自分の見ていた世界に怯えるな
 どあってはならないことだ。改めて、あの頃見ていた世界を描くしかないと決
 意した。これは私にとっての試練なのだと思う。\\

 (省略)\\

 \tt{2010年 11月20日}

 筆を握るたびに冷や汗をかく。やはり、あのような世界を思い出すのは私にとっ
 て辛いものがある。しかし、そうは言っていられない。ひと塗りするごとに、
 この試練を乗り越えている実感があるのも事実なのだ。\\

 (省略)\\

 \tt{2011年4月1日}

 ついに絵が完成した。あまりに今までの私の絵とは方向性が違いすぎるので、
 売れるとは思えない。しかし、これでよかったのだ。私はまたひとつ、自分自
 身を乗り越えることができた。\\

 (省略)\\
\end{itembox}

二階に上ると、壁や絨毯に染み付いた絵の具の匂いが漂っている。中央には大き
な部屋があり、イーゼルが並んでいる。外山はここをアトリエとして使っていた
ようだ。あたりを見渡すと、以下のようなものが見つかる。

\begin{topic}
 \item[書きかけの絵]
 窓から見えるプールが赤や緑で塗りたくられた絵がある。なかなか納得が行か
 なかったのか、何度も何度も重ねて塗り直した形跡が見られる。
 不気味な色彩の中に狂気を感じ、寒気がした。SANチェック[0/1]。

 \item[外山の日記2]
 書斎から抜けていた最新の日記が、画材が積んである机の上においてあるのが
 見つかった。初めのほうこそ昨年のものと変わらない内容だが、ある日を境に
 外山に変化があったのがわかる。
\end{topic}

\begin{itembox}{外山の日記2}
 \tt{2011年 7月22日}

 三船くんによると、どうやら新しい私の絵は好評らしい。今まで美しいと思っ
 て書いていた絵よりも、恐ろしいと思って書いた絵のほうが売れるというのは
 皮肉なものだ。\\
 しかし、もはやこの色、この光を恐ろしいと思わなくなっている私がいるのも
 事実だ。この世界こそが私の原点なのだろう。きっと初めてあの光を見た
 時に感じた懐かしさも、これに起因するものだったに違いない。\\

 (省略)\\

 \tt{2011年 7月30日}

 真夜中のことだった。最近はやせ細って元気がなかったケリーがやたらと吠え
 ているので起きてみたら、プールがあの何とも言えない、懐かしい色に光り輝
 いていた。あの日、空から降ってきた何かは、まだあの底に沈んでいたのだ。\\
 この光景を目に焼き付けておこう。次に書く絵はこれに決めた。\\

 (省略)\\

 \tt{2011年 8月3日}

 どうしてもあの色が再現できない。自分の技術の足りなさを実感する。どんな
 手段を用いてもこの絵だけは完成させたい。たとえ、この絵が私の最後の作品
 になろうともだ。\\

 \tt{2011年 8月8日}

 この日記には書いていなかったが、先月頃から体調が芳しくない。あまり私も
 長くはないのだろう。なぜかは分からないが、感覚的に死期が近いのを感じる。
 \\
 どうしても生きているうちにこの絵を描き上げねば、死んでも死にきれない。
 かくなる上はもう一度、あの色を見に行くしか方法はない。\\

 (以降白紙ページ)

\end{itembox}

ここまで日記を読んでしまったPCはアイデアロール。

\begin{judge}{アイデア}
 \item 切羽詰まった外山が「あの色」を見るため、プールへ潜ろうとしたこと
 に気づき、つい窓から外を見てしまう。
\end{judge}

藻が生い茂ったプールの表面に、灰色になったボロ布のような遺体が浮かんでい
る。SANチェック[1D2/1D4]。

\subsubsection{遺体発見後}
警察に通報してしばらくすると、最寄りの警察署からパトカーなどが何台か到着
した。 現場検証に立会い、ここに至った経緯、発見時の状況などを根掘り葉掘
り聞かれた。\\

「遺体を見つけた時の様子を教えてもらえるかな。どうして死体があることに気
づいたんだい?」

「で、君等はここで何をしていたんだ?」\\

など、必要に応じて信用、言いくるめで判定を。

解放されて宿に戻った頃には17時を回っていた。日が沈みはじめ、街は不気味な
ほどに静まり返っている。\\

死体発見の話が広まったのだろう、女将が「大変でしたね、今日はゆっくりお休
みください」と言って気を使ってくれた。\\

(適当にロールプレイ)\\

警察から電話があり、「プールを\ruby{浚}{さら}うが、何か出るかもしれないから
立ち会ってくれ」とのこと。9時頃に迎えが来ることに。

\subsubsection{現場検証に立会い}
指定の時間にパトカーが迎えに来た。車に乗って外山邸へ向かう。\\

現場に到着すると、昨日の刑事がいる。彼の話によると、服についた絵の具の後
などから、おそらく外山昭三本人であることに間違いはないと断定された。また、
水に濡れて壊れた腕時計から、最後の日記を書いた次の日にプールに入ったもの
と思われる。死因は不明で、なぜ腐敗しなかったのか、あのような状態になった
のかも含めて全くわからないという。鑑識も「今までこんな遺体は見たことがな
い」と頭を抱えている。\\

とりあえず今日はプールを漁って何か手がかりになりそうなものが出ないかを確
かめるらしいが、彼はあまり大したものが出るとは思っていないようだ。\\

その時、プールの方へ向かった警官たちから悲鳴が上がった。\\

そちらへ向かうと、プールの周りや水面で、おびただしい量のカラスが死んでい
るのが見える。どれも昨日の遺体のように灰色に\ruby{萎}{しぼ}んでいる。SAN
チェック[1D2/1D4]。

追加でアイデアロールに成功すると、昨日の夜、カラスが全く鳴いていなかった
ことに気づく。おそらく警察が引き上げた後でカラスが集結し、何らかの原因で
一斉に死んだのだろう。\\

それでも警察は仕事なので、カラスが死んでいるくらいでは引き上げる訳にはい
かない。三人の警官が潜水服に着替えて濁った水の中の探索が始まった。\\

しばらくして、同じように灰色になった犬の死体が引き上げられた。外山の絵に
描かれていたのと同じ犬だと思われるが、もはや見る影もない。\\

\begin{judge}{聞き耳}
 \item ほぼ無風であるにもかかわらず、なぜか木々がざわめいていることに気づく。
\end{judge}

そうして探索が進むうちに、捜査員に異変が現れた。プールの中央辺りで潜って
いた捜査員が。酸素ボンベをつけているにもかかわらず苦しそうに水面付近でも
がいている。気づくとプールが淡く発光している。まさに、あの未完成の絵に描
かれていたプールのように。SANチェック[1d3/1d6]。\\

プールに飛び込んで捜査員を助けることもできる。飛び込んだ瞬間に、体から力
が抜けていくのがわかる。体が重くなり、いつもの様に動けない。SANチェック
[1d2/1d4]。\\

\begin{judge}{水泳}
 \item どうにか溺れている捜査員を引き上げることに成功した。耐久力に-2。
 \item 体がうまく動かず、思うように岸へ進めない。耐久力に-2してもう一回
 水泳ロール。
\end{judge}

4回目までに成功できれば、捜査員はなんとか生き\ruby{存}{ながら}えるが、
5回目に突入してしまうと死亡する。その場合、助けに行ったPCは生還後にSAN
チェック[1d2/1d4]。逆に助けられた場合、死地を乗り切ったということでSANを
1D6回復して良い。

しばらくするとプールの光が強くなり、木々のざわめきも大きくなっていく。目
を覆わんばかりのギラギラした輝きが極大に達したと同時、プールの中に潜んで
いたと思われる何かは轟音とともに空へと登っていった。しばらくは「何か」が
登っていった奇跡に光の柱が残っていたが、それも時間とともに消えていった。


\newpage
\subsection{エピローグ}
あのあと、警察は完全にプールの底を漁って何とか水をすべて抜いたが、あの禍々
しい光に関するものは何一つ得ることができなかったという(もし捜査員を救出
しなかった場合、捜査員が死亡したことが伝えられる)。\\

また、早希に外山が死亡していたことを伝えると、言葉に詰まりながらも礼を述
べた。涙をこらえる彼女を慰めていた都筑によると、葬儀は唯一外山と親交があっ
た早希が執り行うことになったようだ。\\

また、今回の謝礼として、外山昭三の初期の作品で、外山邸近くの海岸を描いた
絵が送られた。これは都筑の一存でオカ研に飾られる事になった。\\

余談ではあるが、ついに外山が完成させることができなかった絵は、彼の遺作と
して、ファンの間で大層高値で取引されたという。\\

もちろん、あの天へと登っていった光については何もわからないままであった。
\\


\begin{flushright}
 ---------「懐かしい色」 了
\end{flushright}

\newpage
\section{付録}
\subsection{大月町への経路例}
東京から最寄り駅である宿毛までの経路。宿毛から大月町まではバスで20分ほど。
\begin{screen}
\begin{verbatim}
発着時間:08:18発 → 14:00着
所要時間:5時間42分
乗車時間:4時間34分
乗換回数:5回
総額:40,310円

■東京    5番線発
|  山手線品川方面行   3.1km   7・10号車
|  08:18-08:24[6分]
|  150円
◇浜松町    3番線着 [5分待ち]
|  東京モノレール空港快速(羽田空港第2ビル行)   17.0km   
|  08:29-08:47[18分]
|  470円
◇羽田空港第1ビル/羽田空港     [38分待ち]
|  JEX1487便   632.0km   
|  09:25-10:45[80分]
|  33,770円( 片道 )
◇高知空港     [10分待ち]
|  高知空港線[高知](朝倉[高知大学前]行)      
|  10:55-11:30[35分]
|  700円
◇高知駅/高知    1番線発 [9分待ち]
|  南風3号(中村行)   115.1km   
|  11:39-13:24[105分]
|  2,960円( 指定席 2,260円 )
◇中村     [6分待ち]
|  土佐くろしお鉄道宿毛線(宿毛行)   23.6km   
|  13:30-14:00[30分]
|   ↓ 
■宿毛
\end{verbatim}
\end{screen}
\newpage

\section{ハンドアウト}
\subsection{外山の日記 2010年}
\begin{screen}
 
 \tt{2010年 11月11日}
 
 今日は朝から空模様が怪しかったが、昼過ぎになって雨がふりだした。\\
 夕方の五時頃だっただろうか、庭ですごい音がしたので振り向くと、強烈な光
 に目を焼かれた。今まで見たことのないような色の光だったが、何故か懐かし
 い感じがした。いったい何だったのだろうか。\\
 プールの水面がひどく荒れていたので、隕石か何かが落ちてきたのかもしれな
 い。\\

 \tt{2010年 11月12日}

 今日は一日、あの光について考えていたが、ようやくあの懐かしい感覚の正体
 を掴んだと思う。\\
 私はかつて、あのような色だけを見て過ごしていたのだ。完全に同じ色であっ
 たかどうかは定かではないが、近いものがあったのだと思う。\\

 \tt{2010年 11月13日}

 どうもあの光を見てからというもの、何か落ち着かないまま時間だけが過ぎて
 いく。どうしても、かつて私が見ていた世界が思い起こされるのだ。今描いて
 いる絵も捗らない。\\
 ケリーの様子がおかしいのも気にかかる。いつもはもっと落ち着いているのだ
 が、昨日からずっとソワソワしているようだし、急に焦ったように吠え始める
 のも心配だ。今までこんなことはなかった。\\

 \tt{2010年 11月14日}

 やはりこのままではいけない。画家たるもの、自分の見ていた世界に怯えるな
 どあってはならないことだ。改めて、あの頃見ていた世界を描くしかないと決
 意した。これは私にとっての試練なのだと思う。\\

 (省略)\\

 \tt{2010年 11月20日}

 筆を握るたびに冷や汗をかく。やはり、あのような世界を思い出すのは私にとっ
 て辛いものがある。しかし、そうは言っていられない。ひと塗りするごとに、
 この試練を乗り越えている実感があるのも事実なのだ。\\

 (省略)\\

 \tt{2011年4月1日}

 ついに絵が完成した。あまりに今までの私の絵とは方向性が違いすぎるので、
 売れるとは思えない。しかし、これでよかったのだ。私はまたひとつ、自分自
 身を乗り越えることができた。\\

 (省略)\\
\end{screen}
\pagestyle{empty}
\newpage
\subsection{外山の日記 2011年} 
\begin{screen}
 \tt{2011年 7月22日}

 三船くんによると、どうやら新しい私の絵は好評らしい。今まで美しいと思っ
 て書いていた絵よりも、恐ろしいと思って書いた絵のほうが売れるというのは
 皮肉なものだ。\\
 しかし、もはやこの色、この光を恐ろしいと思わなくなっている私がいるのも
 事実だ。この世界こそが私の原点なのだろう。きっと初めてあの光を見た
 時に感じた懐かしさも、これに起因するものだったに違いない。\\

 (省略)\\

 \tt{2011年 7月30日}

 真夜中のことだった。最近はやせ細って元気がなかったケリーがやたらと吠え
 ているので起きてみたら、プールがあの何とも言えない、懐かしい色に光り輝
 いていた。あの日、空から降ってきた何かは、まだあの底に沈んでいたのだ。\\
 この光景を目に焼き付けておこう。次に書く絵はこれに決めた。\\

 (省略)\\

 \tt{2011年 8月3日}

 どうしてもあの色が再現できない。自分の技術の足りなさを実感する。どんな
 手段を用いてもこの絵だけは完成させたい。たとえ、この絵が私の最後の作品
 になろうともだ。\\

 \tt{2011年 8月8日}

 この日記には書いていなかったが、先月頃から体調が芳しくない。あまり私も
 長くはないのだろう。なぜかは分からないが、感覚的に死期が近いのを感じる。
 \\
 どうしても生きているうちにこの絵を描き上げねば、死んでも死にきれない。
 かくなる上はもう一度、あの色を見に行くしか方法はない。\\

 (以降白紙ページ)

\end{screen}

\end{document}

