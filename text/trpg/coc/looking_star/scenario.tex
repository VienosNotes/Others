\documentclass[a4paper,8pt,min]{jsarticle}
\usepackage[dvipdfmx]{graphicx}
\usepackage{txfonts}
\usepackage{trpgmacro}
\usepackage{pxrubrica}
\usepackage{fancybox}

\title{{\large{クトゥルフ神話TRPGシナリオ}}\\ 星を見る少女
}
\author{VienosNotes}

\usepackage{ascmac}	% required for `\itembox' (yatex added)
\begin{document}
\maketitle
\thispagestyle{empty}
\newpage
\setcounter{page}{1}
\section{シナリオ概要}
筑波大学に長年受け継がれてきた都市伝説「星を見る少女」。ある日を境にそ
の噂が急に広がり始め、ついには怪我人が出るようにまでなった。

それを知ったオカ研こと都市文化研究会の顧問である都筑稔は「嫌な予感」を
いだき、自体の収束をオカ研の部員たちに託す。

\section{設定}
\subsection{時期}
2011年10月18日〜シナリオ終了まで。

\subsection{舞台}
国立大学法人筑波大学。立地・構造は現実の筑波大学と同じ。

\subsection{登場人物}
\subsubsection{\ruby[m]{都筑稔}{つ|づき|みのる}}
51歳、男性。比較文化学類の教授。自分の研究室を「オカルト研究会」こと都市文化
研究会として気に入った学生に使わせており、また自身もオカ研の一員として面
白そうなことに首を突っ込む。民俗学や都市伝説について研究をしており、そち
らの専門家としても教鞭を振るう。

\subsubsection{\ruby[m]{坂田真司}{さか|た|しん|じ}}
19歳、男性。応用理工学類の学生。「星を見る少女」と遭遇し、病院に運ばれた
張本人。気が弱い。

\subsubsection{\ruby[m]{岡野彰}{おか|の|あきら}}
42歳、男性。生物資源学類で准教授を務める。20年前に交際中だった女性を自殺
で失っている。

\subsubsection{\ruby[m]{豊橋菜穂}{とよ|はし|な|ほ}}
享年19歳、女性。岡野と交際していたが、暴行事件に巻き込まれひどく傷つき自
殺。

\newpage
\section{シナリオ本編}
\subsection{導入 --- 廊下}
舞台は茨城県にある筑波大学、時期はそろそろコートがないと肌寒くなってきた10月。
PLたちはサークル「都市文化研究会」通称「オカルト研」に所属している。
今日は顧問の都筑稔(つづきみのる)から昼頃に急な連絡があり、放課後は部室に
集まるように言われている。

集合の時間まではまだ少し余裕がある。

\begin{judge}{聞き耳}
 \item トピック「救急車の噂」を得る。
\end{judge}

\begin{topic}
 \item[救急車の噂] そのへんを歩いていた男子学生が次のように会話している
 のが耳に入る。

「いや、俺も詳しくは知らないんだけどさ、昨日平砂で凄い怪我人が出たんだと。
そいつの友達がさ、飯食いに行こうって誘いに行ったら、こう頭から血を流して
 倒れてたって。食事の前に血なんか見たくねえよなぁ、可哀想に」
\end{topic}

そうこうしているうちに、都筑の指定した集合時間が近づいてきた。

\subsection{導入 --- オカ研}
PCたちが部室に到着すると、都筑が窓側でタバコを吸っていた。都筑はドアが開
く音に気づいて、こちらに視線を向ける。

「ああ、急に呼び出してしまって済まないね。どうしても伝えておかなくちゃい
けない用事があったから。そうだ、コーヒーでも飲みながら話そうか」

そう言って都筑はコーヒーを淹れるため、給湯室の方へ行った。\\

この部室は、都筑の研究室を部室と兼用している。
都筑の机には書類が山のように積み上げられ、今にも崩れそうである。

\begin{judge}{目星}
 \item トピック「アメリカ行き」を得る。
 \item トピック「部屋の惨状」を得る。
\end{judge}

\begin{topic}
 \item[アメリカ行き] 明後日からシカゴで開かれる学会の資料などが山の一番
 上に乗っている。前に都筑が学会でアメリカに行くと言っていたことを思い出
 した。
 \item[部屋の参上] 部屋の至る所に、得体の知れない置物や鏡などが飾ってある。
\end{topic}
\newpage

お盆にコーヒーを載せた都筑が戻ってきた。\\
「それで今日呼んだ理由なんだけれども。」と話し始める。「昨晩、平砂で何て
言うか------事故があったのは知ってる?」PLに問いかける。\\

(知ってるor知らない、適当にロールプレイ。必要に応じて昨日の騒動の知識を
整理)

「ところで、『星を見る少女』って都市伝説、知ってるかな?」

\begin{judge}{知識,幸運}
 \item トピック「星を見る少女概要」を得る。
 \item 「今の筑波大生は知らないのかな…まあ詳しいことはあとで調べてもらうとして」
\end{judge}

\begin{topic}
 \item[星を見る少女概要] 「星を見てるんだと思ったら首を吊ってる、みたいな話を聞いたことがあったような…」
\end{topic}

「僕も\ruby[m]{人伝}{ひと|づて}に聞いただけなんだけど。最近出るらしいん
だよ、それが」\\

都筑はそう言って部員たちをぐるりと見回す。\\

「考え過ぎだと言われたらそれまでだけど、なんか予感がするんだよね。
 ちょっと自分で調べてみたいのは山々なんだけど…」と肩をすくめる。\\

「(判定の結果によっては)知ってると思うけど、僕は明後日から始まる学会に行くから、
明日には日本を発つ。残念ながら、僕には調査をしている時間はない。------も
しよかったら、君らに代役を頼めないかな」\\

(何とかPCに依頼を呑んでもらう)\\

「ありがとう。僕の思い過ごしだと良いんだけどね。まずは宿舎に住んでる人に
話を聞くのが良いんじゃないかな。結構噂になってるみたいだし」\\
そう言って、都筑は机の上の書類を鞄に詰め込み始めた。

\subsection{調査開始 --- 平砂宿舎}
時刻は午後18時過ぎ、風呂にはいるために歩いている人も何人かいる。詳しい話
を聞くために、声をかけることもできる。

\begin{judge}{幸運/30min.}
 \item 成功するごとに以下のトピックを1つずつ得る。
\end{judge}

\begin{topic}
 \item[目撃情報] 「同じ学年の娘から聞いた話なんだけど…」と前置きして話
 し始める。
「二三日前、どこだったかの部屋の窓から空を見上げている女の子が見えてたん
 だって。『星を見る少女』の話もその娘から聞いたんだけど、なんか不気味よ
 ねぇ」

 \item[]
\end{topic}

\end{document}
