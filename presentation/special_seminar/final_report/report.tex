\documentclass[a4j,9pt,titlepage]{jsarticle}
\usepackage{stdrep}
\usepackage{ascmac}	% required for `\screen' (yatex added)
\usepackage{fancybox}


\geometry{left=30mm,right=30mm,top=30mm,bottom=35mm}

\setstretch{1.15}

\title{情報特別演習 最終レポート \\ 「TeXコンパイルサーバの作成」}
\author{学籍番号 200911434 \\ 名前 青木大祐}

\begin{document}
\maketitle
\section{背景}
文書や論文作成用のツールとして、広く用いられているソフトウェアのひとつに
TeXが挙げられる。このツールはテキストベースで綺麗な書類や数式を記述でき、
またプログラミング言語のようにマクロを記述できるなど高い拡張性を持ってい
る。しかし、このTeXには幾つかの問題がある。

まず、環境構築の手間が大きい点が挙げられる。例えばMacOS XにTeXを導入する
ためのMac TeX\footnote{http://tug.org/mactex/}パッケージのサイズは2GBを超えており、ダウンロードするだけで
非常に長い時間がかかる。また、複数のコンピュータでスタイル(.styファイル
やフォント)を共有する際に問題が起きることがあり、文書を作成したコンピュー
タではコンパイルが出来たはずなのに、他のコンピュータでコンパイルしようと
した時にスタイルが適用できなくて失敗することがある。

次に、文法が難解であることも問題である。標準で用意されている機能を使って
いるうちは良いが、自分でマクロを記述しようとすると、複雑なTeX言語の仕組みに
ついての学習が必須となる。また、箇条書き---itemizeなど高頻度に使用する命令であって
も記述量がかなり多く、ワープロソフトを用いればボタンひとつで
完了できる操作であっても、TeX言語では幾つもの命令を入力する必要がある。\\

こうした問題を抱えているTeXであるが、普段から利用しているソフトウェアだ
けに、代替品を探すというのも難しい。よりTeXを使いやすくするために、これ
らの問題点のうち、幾つかの点について改善を試みた。

\section{方針}
まず、環境構築についてのコストを軽減するために、実際に動作するTeXのシス
テムを1台のサーバに集約することを考える。これによって、スタイルファイル
の共有についても同じく解決されることが期待される。サーバにはブラウザから
アクセスできるWeb UIを実装し、TeXがインストールされていないコンピュータ
からはブラウザを介してTeXの機能を利用できるようにする。

\begin{figure}[h]
 \begin{center}
  \includegraphics[width=6cm]{integrate.eps}
 \end{center}
\end{figure}

次に、TeX言語の記述量の多さを軽減するために、簡略な表記の導入を考える。これ
はTeX言語のレイヤでマクロを実装するのでは無く、TeXに入力を与える前段階で
のプリプロセスとしてTeX言語に変換する。TeX言語のように構造を持ったテキストベース
の記法のうち、メジャーであるMarkdown記法とTextile記法の2つを、代替として、またTeX
言語と混在して記述できるようにする。

\section{実装}
前章の方針を基に、問題点を解決するためのプログラムを実装した。このプログ
ラムは大きく分けて2つの部分に分かれており、サービスを提供するコンピュー
タで動作する部分を{\bfseries サーバ}、サービスを利用するコンピュータで動
作する部分を{\bfseries クライアント}と呼ぶことにする。

以下に、システムの概要を図示する。\\
\begin{figure}[h]
\begin{center}
 \includegraphics[width=8cm]{system.eps}
\end{center} 
\end{figure}

制作に際して、サーバ側にはスクリプト言語Perlを、クライアント側には
HTML5+JavaScriptを用いて実装を行った。

\subsection{サーバ}
サーバ部分の処理は、次のような流れで行われる。
\begin{enumerate}
 \item ブラウザからのリクエストに応答してWeb UI(クライアント)を返す
 \item クライアントに入力されたソースコードをAjaxで受け取る
 \item 簡略表記が使われていれば、ルールに従ってTeX言語に変換する
 \item ソースコードをTeXに渡し、DVIを生成する 
 \item 生成したDVIをクライアントに返す
 \item ダウンロード要求を受け取り、PDFを生成して返す
\end{enumerate}
これらを実現するために、スクリプト言語Perlで用いられる軽量
WAF\footnote{Web Application Framework}である
Amon2\footnote{http://amon.64p.org/}を用いて実装を行った。Amon2はPSGI
アプリケーションを記述するための軽量フレームワークであり、少ない記
述量で高機能かつ安全なアプリケーションが制作できる。同じくPerlのWAFとし
て広く用いられているCatalystに対してシンプルな仕組みが特色であり、学習に
かかるコストも比較的小さい点も魅力である。\\

以下に、各ステップごとの具体的な処理の内容を説明する。
\subsubsection{リクエストへの応答}
ユーザがTeX言語などのソースコードを入力する画面をWebサーバとして提供する。
クライアントの詳しい実装については次節で述べる。

\subsubsection{ソースコードを受信}
Ajaxでソースコードを受信するためのURLでPOSTを受け取り、ファイルとして保
存する。保存先のディレクトリは他のファイルと衝突しないように一意な名前が
振られるようになっており、URLを保存しておくことで任意のタイミングでファ
イルが参照できる。

\subsubsection{形式の変換}
簡略表記としてMarkdown記法またはTextile記法が使われている場合は、これを
TeX言語に変換する。全体が簡略表記で記述されている場合は、ドキュメント変換
ツールのPandoc\footnote{http://johnmacfarlane.net/pandoc/}を用いて変換を
行う。

また、TeX言語に埋め込んで簡略表記が使われている場合はPandocが使えないため、
自作の変換ツールを用いて変換を行う。これはTeX言語中の任意の場所で簡略表
記が使える代わりに、Pandocを用いるパターンよりも機能が少なくなっている。

\subsubsection{DVIの生成}
生成されたTeX言語のソースファイルを、TeXの処理系のひとつであるplatexを用
いてDVIに変換する。TeXの命令としてOSのコマンドを実行するものがあり、その
ままでは危険なので予め使えないように設定しておく必要がある。

\subsubsection{DVIをクライアントに返す}
送られてきたリクエストへのレスポンスとして、生成されたDVIを返す。この段
階でPDFを生成しない理由としては、PDFよりもDVIのほうがファイルサイズが小
さいため、送信にかかる時間が節約できることが挙げられる。

\subsubsection{PDFのダウンロード}
ダウンロード用のURLにリクエストを受け取ると、dvipdfmxを用いてPDFを生成し、
これをユーザに返す。

\subsection{クライアント}
次に、クライアントの実装について述べる。

クライアントはブラウザで表示するため、HTML5を用いて実装している。ページ
デザインには、Amon2にバンドルされているCSSライブラリ、Twitter
Bootstrap\footnote{http://twitter.github.com/bootstrap/}
を使用している。

クライアント側の処理の流れは以下のようになる。
\begin{enumerate}
 \item ページを表示する
 \item ユーザの入力をAjaxでサーバに送信する
 \item 帰ってきたDVIを表示する
 \item 内容に問題が無ければPDFをダウンロードする
\end{enumerate}

以下に、各ステップごとの具体的な処理の内容を説明する。

\subsubsection{ページを表示する}
サーバからページを受信し、ブラウザに表示する。具体的なページのレイアウト
に関しては、後述する。

\subsubsection{入力を送信}
用意された送信用のURLに対して、フォームの内容などを非同期通信でPOSTする。

\subsubsection{DVIの表示}
サーバから返ってきたDVIを、
dvi.js\footnote{https://github.com/naoyat/dvi.js/wiki}を用いてページ内に
インライン表示する。実装当初は生成したサムネイル画像を表示する設計であっ
たが、ファイルサイズの問題からどうしても受信に時間が掛かってしまうため、
JavaScriptでDVIを表示するdvi.jsを採用することにした。この場合、受信から
描画に掛かる時間は、もはやユーザがストレスとして感じない程度に削減するこ
とができる。また、ファイルサイズとの兼ね合いで低解像度で表示せざるを得な
い画像と比べて、ブラウザに表示されるページの一部として描画されるため、
dvi.jsを用いて描画した場合のほうが綺麗に表示されるという利点もある。

実際にdvi.jsによってDVIが描画されているスクリーンショットを以下に示す。
ページ中央のテキストボックスにソースコードを入力し、下部のコンパイルボタ
ンをクリックすることで、最下部にDVIプレビューが表示される。
\\

\begin{figure}[H]
 \begin{center}
  \includegraphics[width=10cm]{ss.png}
 \end{center}
\end{figure}
\subsubsection{PDFのダウンロード}
ダウンロード用のURLにアクセスすることで、PDFをダウンロードできる。\\

以上が、サーバとクライアントの実装の概要である。
\section{考察}
これらの機能を実装することで、TeXの環境構築に掛かる時間的コストは削減す
ることが可能になったと考える。また、簡略表記の導入によって学習コストを
軽減し、非利用者に対する入門の敷居を下げることにも繋がると思う。

\section{今後の課題}
今回TeXコンパイルサーバを実装して、足りないと思った機能がいくつかあり、
これらは今後の課題として逐次実装していけたら良いなと思っている。\\

まず、ユーザ個人用のファイルを保存する機能が必要である。講義の課題などの
シンプルな書類であれば問題ないこともあるが、この報告書のようなレポー
トや論文などにおいて、画像が挿入できないのは致命的である。

また、エディタが貧弱なのも課題といえる。これは命令の挿入など入力を支援
する機能や、シンタックスハイライトをクライアント側に実装することで解決さ
れると考えられる。

最後に、私のように普段からTeXを使うユーザにとっては、コマンドラインから
操作できたほうが都合の良いことは多い。そのため、ブラウザだけでなくコマン
ドラインからのリクエストを送信できるようなソフトウェアも作成したいと思う。

\section{感想}
もともと私はPerlを主に用いていたのだが、今ままでWebアプリケーションを書
いたことは無かった。しかしCGIで一大勢力を築いただけあって、PerlのWebにお
ける強みは周知の事実である。折角Perlを使っていながらWebアプリケーション
を書かないというのはもったいないという事で、今回の演習にはPerlを用いて実
装を行なってみた。その結果、Perlという言語の柔軟性、また拡張性の高いフレー
ムワークの強力さが実感できたように思う。

また、クライアント部分に用いたJavaScriptは私の苦手な言語の筆頭であった。し
かし、ある程度の量を書くと、一見奇妙に思える言語の特性が、ブラウザ上で動
作する言語として大きな強みになっている部分もあるのだと分かった。この演習
を通して苦手意識が薄れたのは大きな収穫だったと思う。\\

なお、今回の演習で作成したソフトウェアはGithubにて公開\footnote{https://github.com/VienosNotes/DominoTeX}してい
る。

\end{document}
