\documentclass[14pt,dvipdfm,trans]{beamer}
% pdfの栞の字化けを防ぐ
% テーマ
\usetheme{CambridgeUS}
% navi. symbolsは目立たないが,dvipdfmxを使うと機能しないので非表示に
\setbeamertemplate{navigation symbols}{} 
\usepackage{graphicx}
\usepackage{amsmath}
\usepackage{amssymb}
% フォントはお好みで
\usepackage{txfonts}
\usepackage{strike}
\mathversion{bold}
\renewcommand{\familydefault}{\sfdefault}
\renewcommand{\kanjifamilydefault}{\gtdefault}
\setbeamerfont{title}{size=\large,series=\bfseries}
\setbeamerfont{frametitle}{size=\large,series=\bfseries}
\setbeamertemplate{frametitle}[default][center]
\usefonttheme{professionalfonts} 

\usepackage{listings}


\title{TeXサーバの制作}
\author{青木大祐}
\institute{情報科学類}
\date{情報特別演習 中間発表}

\usepackage{color}	% required for `\textcolor' (yatex added)
\begin{document}
\frame{\titlepage}

\begin{frame}
 \frametitle{やりたいこと}
 \begin{itemize}
  \item WebからTeXを利用できるWebサービスを実装
        \begin{itemize}
         \item マシンごとに入ってるライブラリとかが違ってて面倒
         \item そもそもTeXが大きすぎて入れるのに時間が掛かる
        \end{itemize}
        \vspace*{1zh}
  \item 単純なマークアップ言語でも記述できるように
        \begin{itemize}
         \item TeXは難解過ぎる
         \item 大体の場合はもっと単純な命令だけで充分
         \item とにかく簡単にレポートの体裁を整えたい
        \end{itemize}
 \end{itemize}
\end{frame}

\end{document}