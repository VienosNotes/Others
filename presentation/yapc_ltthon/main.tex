\documentclass[14pt,dvipdfm,trans]{beamer}
% pdfの栞の字化けを防ぐ
% テーマ
\usetheme{CambridgeUS}
% navi. symbolsは目立たないが,dvipdfmxを使うと機能しないので非表示に
\setbeamertemplate{navigation symbols}{} 
\usepackage{graphicx}
\usepackage{amsmath}
\usepackage{amssymb}
% フォントはお好みで
\usepackage{txfonts}
\usepackage{strike}
\usepackage{color}
\mathversion{bold}
\renewcommand{\familydefault}{\sfdefault}
\renewcommand{\kanjifamilydefault}{\gtdefault}
\setbeamerfont{title}{size=\large,series=\bfseries}
\setbeamerfont{frametitle}{size=\large,series=\bfseries}
\setbeamertemplate{frametitle}[default][center]
\usefonttheme{professionalfonts} 

\usepackage{listings}


\title{日本でPerl6を学びたい}
\author{@VienosNotes/青木大祐}
\institute{Tsukuba.pm}
\date{YAPC::Asia Tokyo 2012 LT-thon}

\usepackage{color}	% required for `\textcolor' (yatex added)
\begin{document}
\frame{\titlepage}

\begin{frame}
 \frametitle{はじめに}
 \begin{itemize}
  \item Perl6を追い始めたのは二年半くらい前
\vspace{1zh}
  \item いろんなバグを踏みつつ独学である程度は使えるようになった
 \end{itemize}
\vspace{2zh}
→ 結構\textcolor{red}{苦難}の道だった
\end{frame}

\begin{frame}
\frametitle{何が困るって}
\begin{itemize}
 \item 日本語のドキュメントが皆無
 \item 英語のものも古い情報が残ってて騙される
 \item 誰も使ってないから教えてくれる人が(ほとんど)いない
\end{itemize}
\end{frame}

\begin{frame}
 \frametitle{そもそも}
\begin{center}
 \large{日本にPerl6のコミュニティとかあんの?}
\end{center}
\end{frame}

\begin{frame}
 \begin{center}
  \huge{探したけど無いっぽい} 
 \end{center}
\end{frame}

\begin{frame}
\begin{center}
 \Huge{まぁ作ればいいか} 
\end{center}
\end{frame}

\begin{frame}
 \frametitle{make community}
\begin{itemize}
 \item 筑波大学には奇跡的にPerl6を追いかけてる人がもう一人居た!
       →\textcolor{red}{誘致}
\vspace{2zh}
 \item 知り合いにPerlerは結構いる →\textcolor{red}{誘致}
\end{itemize}


\end{frame}

\begin{frame}
\begin{center}
 メンバーわずか5人の「仕様書読書会」が発足\\
\vspace{1zh}
\Large{→日本で唯一のPerl6勉強会}
\end{center}
 
\end{frame}

\begin{frame}
 \frametitle{その後}
\begin{itemize}
 \item 何度かの読書会を経て、読みたかった部分は目を通し終えた
 \item その後は不定期に開催、新しく入った面白い機能の紹介など
\end{itemize}
\end{frame}

\begin{frame}
  \begin{center}
  \Large{去年はPerl6 Advent Calendarを完走!}
 \end{center}

\end{frame}

\begin{frame}
\frametitle{今年の新しい試み}
\begin{itemize}
 \item ビギナー向けのスタートアップ
 \begin{itemize}
        \item Larry Wallの講演で興味を持った人
        \item 過去にドキュメントが無くて諦めた人
       \end{itemize}
\vspace{1zh}
 \item 11月を目標に開催
       \begin{itemize}
        \item そもそもの人数が少ないので、参加したい人の都合にあわせて開
              催できる!
       \end{itemize}
\end{itemize}
\end{frame}

\begin{frame}
 \begin{center}
  ただし筑波まで来てもらう必要があります\\
  \vspace*{3zh}
  \tiny{参加人数によっては都内での開催も検討するかも…?}
 \end{center}
\end{frame}

\begin{frame}
 \frametitle{最終的な目標というか個人的な野望}
\begin{center}
 \Large{日本語ドキュメントの整備}\\
 \vspace*{1zh}
 \small{ぶっちゃけて言うと}\\
 \vspace*{1zh}
\huge{Perl6の本が書きたい}
\end{center}
\end{frame}

\begin{frame}
\vspace*{1zh}
 ということで興味を持ってくれた方は\\
\begin{center}
 \huge{@VienosNotes}\\
\vspace*{0.5zh}
\end{center}
\begin{flushright}
 \normalsize{までよろしくお願いします。} 
\end{flushright}
\end{frame}
\end{document}